% TPCU 碩士學位論文 LaTex 樣板
% 使用 utf-8 編碼
% v0.5 (Jun. 30, 2012)
% by Peter

\documentclass[a4paper,12pt]{report}


% 引用必要的套件
\usepackage{tpcu}
\usepackage[english]{babel}
\usepackage{blindtext}
% Too many unprocessed floats
\usepackage{etex}
\reserveinserts{18}
\usepackage{morefloats}
%\usepackage{chap_last}

\usepackage{xcolor,tikz}
%\usepackage[usenames,dvipsnames,pdftex]{xcolor}
%\def\pgfsysdriver{pgfsys-dvipdfmx.def}
%\usepackage{tikz,ifthen}

\newcommand{\newchapter}[1]{%
    \chapter{#1}
%   \pagestyle{fancy}
    \thispagestyle{fancy}
%   \fancyhead[LO,RE]{\slshape \leftmark}
    \fancyhead[LE,RO]{\slshape \leftmark}
    \fancyfoot[C]{\thepage}
%   \fontsize{16}{20}\selectfont
}
%\usepackage{CJKnumb} %%% ZZZ %%% for Chinese numbering capability
%\usepackage[nospace]{cite}  % for smart citation

%%%%%%%%%%%%%%%%%%%%%%%%%%%%%
%  end of preamble
%%%%%%%%%%%%%%%%%%%%%%%%%%%%%

% define the title
\begin{document}

\large 
\fontsize{14}{20}\selectfont
%
% 使用 utf-8 編碼
% v2.0 (Apr. 5, 2009)
% do not change the content of this file
% unless the thesis layout rule is changed
% 無須修改本檔內容,除非校方修改了
% 封面、書名頁、中文摘要、英文摘要、誌謝、目錄、表目錄、圖目錄、符號說明
% 等頁之格式

% make the line spacing in effect
\renewcommand{\baselinestretch}{\mybaselinestretch}
\large % it needs a font size changing command to be effective

% default variables definitions
% 此處只是預設值,不需更改此處
% 請更改 my_names.tex 內容
\newcommand\cTitle{論文題目}
\newcommand\eTitle{MY THESIS TITLE}
\newcommand\myCname{王鐵雄}
\newcommand\myEname{Aron Wang}
\newcommand\advisorCnameA{南宮明博士}
\newcommand\advisorEnameA{Dr.~Ming Nangong}
\newcommand\advisorCnameB{李斯坦博士}
\newcommand\advisorEnameB{Dr.~Stein Lee}
\newcommand\advisorCnameC{徐 石博士}
\newcommand\advisorEnameC{Dr.~Sean~Hsu}
\newcommand\univCname{臺北城市大學}
\newcommand\univEname{Taipei Chengshih University of Science and Technology}
\newcommand\deptCname{電子商務研究所}
\newcommand\fulldeptEname{Institute of E-Commerce}
\newcommand\deptEname{Institute of E-Commerce}
\newcommand\collEname{College of Business and Management}
\newcommand\degreeCname{碩士}
\newcommand\degreeEname{Master}
\newcommand\cYear{九十四}
\newcommand\cMonth{六}
\newcommand\cDay{}
\newcommand\eYear{2006}
\newcommand\eMonth{June}
\newcommand\eDay{}
\newcommand\ePlace{Taipei, Taiwan}


 % user's names; to replace those default variable definitions
% 使用 utf-8 編碼
% v2.0 (Apr. 5, 2009)
% 填入你的論文題目、姓名等資料
% 如果題目內有必須以數學模式表示的符號,請用 \mbox{} 包住數學模式,如下範例
% 如果中文名字是單名,與姓氏之間建議以全形空白填入,如下範例
% 英文名字中的稱謂,如 Prof. 以及 Dr.,其句點之後請以不斷行空白~代替一般空白,如下範例
% 如果你的指導教授沒有如預設的三位這麼多,則請把相對應的多餘教授的中文、英文名
%    的定義以空的大括號表示
%    如,\renewcommand\advisorCnameB{}
%          \renewcommand\advisorEnameB{}
%          \renewcommand\advisorCnameC{}
%          \renewcommand\advisorEnameC{}

% 論文題目 (中文)
\renewcommand\cTitle{%
品牌聲望及品牌知覺對消費決策為與消費者滿意度影響之研究-以飾品業 La Jolla公司為例}

% 論文題目 (英文)
\renewcommand\eTitle{
A study of the Influences of Brand perception and Brand Reputation on consumer decision making and Consumer satisfaction – A case study of La Jolla jewelry company}

% 我的姓名 (中文)
\renewcommand\myCname{盧建璋}

% 我的姓名 (英文)
\renewcommand\myEname{jian-Jhang Lu}

% 指導教授A的姓名 (中文)
\renewcommand\advisorCnameA{林慶昌 博士}

% 指導教授A的姓名 (英文)
\renewcommand\advisorEnameA{Dr.~Ching-Chang Lin}

% 指導教授B的姓名 (中文)
\renewcommand\advisorCnameB{}

% 指導教授B的姓名 (英文)
\renewcommand\advisorEnameB{}

% 指導教授C的姓名 (中文)
\renewcommand\advisorCnameC{}

% 指導教授C的姓名 (英文)
\renewcommand\advisorEnameC{}

% 校名 (中文)
\renewcommand\univCname{臺北城市科技大學}

% 校名 (英文)
\renewcommand\univEname{Taipei Chengshih University of Science and Technology}

% 系所名 (中文)
\renewcommand\deptCname{電子商務研究所}

% 系所全名 (英文)
\renewcommand\fulldeptEname{Institute of E-Commerce}

% 系所短名 (英文, 用於書名頁學位名領域)
\renewcommand\deptEname{Institute of E-Commerce}

% 學院英文名 (如無,則以空的大括號表示)
\renewcommand\collEname{College of Business and Management}

% 學位名 (中文)
\renewcommand\degreeCname{碩士}

% 學位名 (英文)
\renewcommand\degreeEname{Master}

% 口試年份 (中文、民國)
\renewcommand\cYear{102}

% 口試月份 (中文)
\renewcommand\cMonth{6}

% 口試日 (中文)
\renewcommand\cDay{26}

% 口試年份 (阿拉伯數字、西元)
\renewcommand\eYear{2013}

% 口試月份 (英文)
\renewcommand\eMonth{June}

% 口試日 (英文)
\renewcommand\eDay{twenty-sixth}


% 學校所在地 (英文)
\renewcommand\ePlace{Taipei, Taiwan}

%畢業級別;用於書背列印;若無此需要可忽略
\newcommand\GraduationClass{100}

%%%%%%%%%%%%%%%%%%%%%%


% 使用 hyperref 在 pdf 簡介欄裡填入相關資料
\ifx\hypersetup\undefined
	\relax  % do nothing
\else
	\hypersetup{
	pdftitle=\cTitle,
	pdfauthor=\myCname}
\fi
	

\newcommand\itsempty{}
%%%%%%%%%%%%%%%%%%%%%%%%%%%%%%%
%       TPCU cover 封面
%%%%%%%%%%%%%%%%%%%%%%%%%%%%%%%
%
\begin{titlepage}
% no page number
% next page will be page 1

% aligned to the center of the page
\begin{center}
% font size (relative to 12 pt):
% \large (14pt) < \Large (18pt) < \LARGE (20pt) < \huge (24pt)< \Huge (24 pt)
%
\makebox[6cm][s]{\Huge{\univCname}}\\  %顯示中文校名
\vspace{1.5cm}
\makebox[12cm][s]{\Huge{\deptCname}}\\ %顯示中文系所名
\vspace{1.5cm}
\makebox[6cm][s]{\Huge{\degreeCname 論文}}\\ %顯示論文種類 (中文)
\vspace{3cm}
%
% Set the line spacing to single for the titles (to compress the lines)
\renewcommand{\baselinestretch}{1}   %行距 1 倍
%\large % it needs a font size changing command to be effective
%\Large{\cTitle}\\  % 中文題目
\fontsize{20}{22}\selectfont{\cTitle} \\
%
\vspace{1cm}
%
%\Large{\eTitle}\\ %英文題目
\fontsize{20}{22}\selectfont{\eTitle} \\

\vspace{4cm}
% \makebox is a text box with specified width;
% option s: stretch
% use \makebox to make sure
% 「研究生:」 與「指導教授:」occupy the same width
\hspace{4.5cm} \makebox[3cm][s]{\Large{研 究 生:}}
\Large{\myCname}  % 顯示作者中文名
\hfill \makebox[1cm][s]{}\\
%
\vspace{1cm}
\hspace{4.5cm} \makebox[3cm][s]{\Large{指導教授:}}
\Large{\advisorCnameA}  %顯示指導教授A中文名
\hfill \makebox[1cm][s]{}\\
%
% 判斷是否有共同指導的教授 B
\ifx \advisorCnameB  \itsempty
\relax % 沒有 B 教授,所以不佔版面,不印任何空白
\else
% 共同指導的教授 B
\hspace{4.5cm} \makebox[3cm][s]{}
\Large{\advisorCnameB}  %顯示指導教授B中文名
\hfill \makebox[1cm][s]{}\\
\fi
%
% 判斷是否有共同指導的教授 C
\ifx \advisorCnameC  \itsempty
\relax % 沒有 C 教授,所以不佔版面,不印任何空白
\else
% 共同指導的教授 C
\hspace{4.5cm} \makebox[3cm][s]{}
\Large{\advisorCnameC}  %顯示指導教授C中文名
\hfill \makebox[1cm][s]{}\\
\fi
%
\vfill
\makebox[10cm][s]{\Large{中華民國\cYear 年\cMonth 月}}%
%
\end{center}
% Resume the line spacing to the desired setting
\renewcommand{\baselinestretch}{\mybaselinestretch}   %恢復原設定
% it needs a font size changing command to be effective
% restore the font size to normal
\normalsize
\end{titlepage}
%%%%%%%%%%%%%%

%% 從摘要到本文之前的部份以小寫羅馬數字印頁碼
% 但是從「書名頁」(但不印頁碼) 就開始計算
\setcounter{page}{1}
\pagenumbering{roman}
%%%%%%%%%%%%%%%%%%%%%%%%%%%%%%%
%       書名頁
%%%%%%%%%%%%%%%%%%%%%%%%%%%%%%%
%
\newpage

% 判斷是否要浮水印?
\ifx\mywatermark\undefined
  \thispagestyle{empty}  % 無頁碼、無 header (無浮水印)
\else
  \thispagestyle{EmptyWaterMarkPage} % 無頁碼、有浮水印
\fi

%no page number
% create an entry in table of contents for 書名頁
\phantomsection % for hyperref to register this
\addcontentsline{toc}{chapter}{\nameInnerCover}


% aligned to the center of the page
\begin{center}
% font size (relative to 12 pt):
% \large (14pt) < \Large (18pt) < \LARGE (20pt) < \huge (24pt)< \Huge (24 pt)
% Set the line spacing to single for the titles (to compress the lines)
\renewcommand{\baselinestretch}{1}   %行距 1 倍
% it needs a font size changing command to be effective
%中文題目
\Large{\cTitle}\\ %%%%%
\vspace{1cm}
% 英文題目
\Large{\eTitle}\\ %%%%%
%\vspace{1cm}
\vfill
% \makebox is a text box with specified width;
% option s: stretch
% use \makebox to make sure
% 「研究生:」 與「指導教授:」occupy the same width
\makebox[3cm][s]{\large{研 究 生:}}
\makebox[3cm][l]{\large{\myCname}} %%%%%
\hfill
\makebox[2cm][s]{\large{Student: }}
\makebox[5cm][l]{\large{\myEname}}\\ %%%%%
%
%\vspace{1cm}
%
\makebox[3cm][s]{\large{指導教授:}}
\makebox[3cm][l]{\large{\advisorCnameA}} %%%%%
\hfill
\makebox[2cm][s]{\large{Advisor: }}
\makebox[5cm][l]{\large{\advisorEnameA}}\\ %%%%%
%
% 判斷是否有共同指導的教授 B
\ifx \advisorCnameB  \itsempty
\relax % 沒有 B 教授,所以不佔版面,不印任何空白
\else
%共同指導的教授B
\makebox[3cm][s]{}
\makebox[3cm][l]{\large{\advisorCnameB}} %%%%%
\hfill
\makebox[2cm][s]{}
\makebox[5cm][l]{\large{\advisorEnameB}}\\ %%%%%
\fi
%
% 判斷是否有共同指導的教授 C
\ifx \advisorCnameC  \itsempty
\relax % 沒有 C 教授,所以不佔版面,不印任何空白
\else
%共同指導的教授C
\makebox[3cm][s]{}
\makebox[3cm][l]{\large{\advisorCnameC}} %%%%%
\hfill
\makebox[2cm][s]{}
\makebox[5cm][l]{\large{\advisorEnameC}}\\ %%%%%
\fi
%
% Resume the line spacing to the desired setting
\renewcommand{\baselinestretch}{\mybaselinestretch}   %恢復原設定
\large %it needs a font size changing command to be effective
%
\vfill
\makebox[4cm][s]{\large{\univCname}}\\% 校名
\makebox[6cm][s]{\large{\deptCname}}\\% 系所名
\makebox[3cm][s]{\large{\degreeCname 論文}}\\% 學位名
%
%\vspace{1cm}
\vfill
\large{A Thesis}\\%
\large{Submitted to }%
%
\large{\fulldeptEname}\\%系所全名 (英文)
%
%
\ifx \collEname  \itsempty
\relax % 沒有學院名 (英文),所以不佔版面,不印任何空白
\else
% 有學院名 (英文)
\large{\collEname}\\% 學院名 (英文)
\fi
%
\large{\univEname}\\%校名 (英文)
%
\large{in Partial Fulfillment of the Requirements}\\
%
\large{for the Degree of}\\
%
\large{\degreeEname}\\%學位名(英文)
%
\large{in}\\
%
\large{\deptEname}\\%系所短名(英文;表明學位領域)
%
\large{\eMonth\ \eYear}\\%月、年 (英文)
%
\large{\ePlace}% 學校所在地 (英文)
\vfill
\makebox[10cm][s]{\Large{中華民國\cYear 年\cMonth 月}}
%\large{中華民國}%
%\large{\cYear}% %%%%%
%\large{年}%
%\large{\cMonth}% %%%%%
%\large{月}\\
\end{center}
% restore the font size to normal
\normalsize
\clearpage
%%%%%%%%%%%%%%%%%%%%%%%%%%%%%%%
%       論文口試委員審定書 (計頁碼,但不印頁碼)
%%%%%%%%%%%%%%%%%%%%%%%%%%%%%%%
%
% insert the printed standard form when the thesis is ready to bind
% 在口試完成後,再將已簽名的審定書放入以便裝訂
% create an entry in table of contents for 審定書
% 目前送出空白頁
\newpage%
%\begin{CJK}{UTF8}{nkai}
\thispagestyle{EmptyWaterMarkPage}  % 無 header,但在浮水印模式下會有浮水印
\phantomsection % for hyperref to register this
\addcontentsline{toc}{chapter}{\nameCommitteeForm}%

\begin{center}
\fontsize{24}{36}\selectfont 碩士學位論文考試委員會審定書\\
\vspace{0.5cm}
\end{center}

\vspace{0.5cm}

%\parbox[s]
%{\cTitle} \\
%{\eTitle} \\

\fontsize{20}{22}\selectfont
本校 \underline{\deptCname} \underline{\myCname}  君\\
所提論文 \uline{\cTitle} \\
%本校 \underline{\deptCname} \underline{\myCname}  君,所提論文 \\
%\fontsize{18}{22}\selectfont\cTitle \\
%\fontsize{20}{22}\selectfont
經本委員會審定通過,合於碩士資格,特此證明。 \\

\vspace{0.1cm}
\begin{tabular}{rp{0.9\textwidth}}
學位考試委員會 \\ & \\
委~員~:
	& \rule{0.6\textwidth}{1pt} \\
	& \\
	& \rule{0.6\textwidth}{1pt} \\
	& \\
	& \rule{0.6\textwidth}{1pt} \\
	& \\
	& \rule{0.6\textwidth}{1pt} \\
	& \\
指~導~教~授~: & \rule{0.6\textwidth}{1pt} \\
	& \\
所~長~: & \rule{0.6\textwidth}{1pt}
\end{tabular}

\vfill
\begin{center}
\makebox[10cm][s]{\Large{中華民國\cYear 年\cMonth 月\cDay 日}}
\end{center}
%\large{中華民國}
%\large{\cYear}% %%%%%
%\large{年}
%\large{\cMonth}% %%%%%
%\large{月}
%\large{\cDay}% %%%%%
%\large{日}\\

%\end{CJK}
\clearpage
\normalsize


%%%%%%%%%%%%%%%%%%%%%%%%%%%%%%%
%       論文口試委員審定書 (計頁碼,但不印頁碼)
%%%%%%%%%%%%%%%%%%%%%%%%%%%%%%%
%
% insert the printed standard form when the thesis is ready to bind
% 在口試完成後,再將已簽名的審定書放入以便裝訂
% create an entry in table of contents for 審定書
% 目前送出空白頁
%%\newpage%
%%{\thispagestyle{empty}%
%%\phantomsection % for hyperref to register this
%%\addcontentsline{toc}{chapter}{\nameCommitteeForm}%
%%\mbox{}\clearpage}

%%%%%%%%%%%%%%%%%%%%%%%%%%%%%%%
%       授權書 (計頁碼,但不印頁碼)
%%%%%%%%%%%%%%%%%%%%%%%%%%%%%%%
%
% insert the printed standard form when the thesis is ready to bind
% 在口試完成後,再將已簽名的授權書放入以便裝訂
% create an entry in table of contents for 授權書
% 目前送出空白頁
\newpage%

%\vspace{3cm}
此頁為「論文授權書」,計頁碼,但不印頁碼,

\vspace{3cm}
在口試完成後,再將已簽名的授權書之後,替換本頁以便裝訂。
{\thispagestyle{empty}%
\phantomsection % for hyperref to register this
\addcontentsline{toc}{chapter}{\nameCopyrightForm}%
\mbox{}\clearpage}

%%%%%%%%%%%%%%%%%%%%%%%%%%%%%%%
%       中文摘要
%%%%%%%%%%%%%%%%%%%%%%%%%%%%%%%
%
\newpage
\thispagestyle{plain}  % 無 header,但在浮水印模式下會有浮水印
% create an entry in table of contents for 中文摘要
\phantomsection % for hyperref to register this
\addcontentsline{toc}{chapter}{\nameCabstract}

% aligned to the center of the page
\begin{center}
% font size (relative to 12 pt):
% \large (14pt) < \Large (18pt) < \LARGE (20pt) < \huge (24pt)< \Huge (24 pt)
% Set the line spacing to single for the names (to compress the lines)
\renewcommand{\baselinestretch}{1}   %行距 1 倍
% it needs a font size changing command to be effective
\large{\cTitle}\\  %中文題目
\vspace{0.83cm}
% \makebox is a text box with specified width;
% option s: stretch
% use \makebox to make sure
% each text field occupies the same width
\makebox[1.5cm][s]{\large{學生:}}
\makebox[3cm][l]{\large{\myCname}} %學生中文姓名
\hfill
%
\makebox[3cm][s]{\large{指導教授:}}
\makebox[3cm][l]{\large{\advisorCnameA}} \\ %教授A中文姓名
%
% 判斷是否有共同指導的教授 B
\ifx \advisorCnameB  \itsempty
\relax % 沒有 B 教授,所以不佔版面,不印任何空白
\else
%共同指導的教授B
\makebox[1.5cm][s]{}
\makebox[3cm][l]{} %%%%%
\hfill
\makebox[3cm][s]{}
\makebox[3cm][l]{\large{\advisorCnameB}}\\ %教授B中文姓名
\fi
%
% 判斷是否有共同指導的教授 C
\ifx \advisorCnameC  \itsempty
\relax % 沒有 C 教授,所以不佔版面,不印任何空白
\else
%共同指導的教授C
\makebox[1.5cm][s]{}
\makebox[3cm][l]{} %%%%%
\hfill
\makebox[3cm][s]{}
\makebox[3cm][l]{\large{\advisorCnameC}}\\ %教授C中文姓名
\fi
%
\vspace{0.42cm}
%
\large{\univCname}\large{\deptCname}\\ %校名系所名
\vspace{0.83cm}
%\vfill
\makebox[2.5cm][s]{\large{摘要}}\\
\end{center}
% Resume the line spacing to the desired setting
\renewcommand{\baselinestretch}{\mybaselinestretch}   %恢復原設定
%it needs a font size changing command to be effective
% restore the font size to normal
\normalsize
%%%%%%%%%%%%%
因近年來隨著時代變遷網際網路盛行,導致網路購物興起源於實體通路商紛紛轉型向虛擬通路,La Jolla 樂活雅鈦鍺精品公司一直保持實體店面的營運模式,但為了增加曝光率,採用網路行銷策略,創立官方網站,經營網路論壇以及加入各大網路通路,期望能提昇品牌知名度。

本研究以 La Jolla 樂活雅鈦鍺精品公司為例,探討公司的經營在導入電子商務經營策略之後,品牌知覺及品牌聲望對消費者決策與消費者滿意度之研究,研究方式採網路問卷進行調查,共獲得599份有效問卷;並使用SPSS 2.0運用敘述統計、信度分析、因素分析、迴歸分析方法,進行研究結果描述與驗證假設。

分析結果得知,品牌知覺與品牌聲望會相互影響,而消費者對於品牌知覺與品牌聲望皆會影響消費者決策與影響消費者的滿意度,而消費的決策也會影響消費者滿意度。

關鍵字:品牌聲望、品牌知覺、消費者行為、消費者滿意度、消費者決策、La jolla樂活雅


%%%%%%%%%%%%%%%%%%%%%%%%%%%%%%%
%       英文摘要
%%%%%%%%%%%%%%%%%%%%%%%%%%%%%%%
%
\newpage
\thispagestyle{plain}  % 無 header,但在浮水印模式下會有浮水印

% create an entry in table of contents for 英文摘要
\phantomsection % for hyperref to register this
\addcontentsline{toc}{chapter}{\nameEabstract}

% aligned to the center of the page
\begin{center}
% font size:
% \large (14pt) < \Large (18pt) < \LARGE (20pt) < \huge (24pt)< \Huge (24 pt)
% Set the line spacing to single for the names (to compress the lines)
\renewcommand{\baselinestretch}{1}   %行距 1 倍
%\large % it needs a font size changing command to be effective
\large{\eTitle}\\  %英文題目
\vspace{0.83cm}
% \makebox is a text box with specified width;
% option s: stretch
% use \makebox to make sure
% each text field occupies the same width
\makebox[2cm][s]{\large{Student: }}
\makebox[5cm][l]{\large{\myEname}} %學生英文姓名
\hfill
%
\makebox[2cm][s]{\large{Advisor: }}
\makebox[5cm][l]{\large{\advisorEnameA}} \\ %教授A英文姓名
%
% 判斷是否有共同指導的教授 B
\ifx \advisorCnameB  \itsempty
\relax % 沒有 B 教授,所以不佔版面,不印任何空白
\else
%共同指導的教授B
\makebox[2cm][s]{}
\makebox[5cm][l]{} %%%%%
\hfill
\makebox[2cm][s]{}
\makebox[5cm][l]{\large{\advisorEnameB}}\\ %教授B英文姓名
\fi
%
% 判斷是否有共同指導的教授 C
\ifx \advisorCnameC  \itsempty
\relax % 沒有 C 教授,所以不佔版面,不印任何空白
\else
%共同指導的教授C
\makebox[2cm][s]{}
\makebox[5cm][l]{} %%%%%
\hfill
\makebox[2cm][s]{}
\makebox[5cm][l]{\large{\advisorEnameC}}\\ %教授C英文姓名
\fi
%
\vspace{0.42cm}
\large{Submitted to }\large{\fulldeptEname}\\  %英文系所全名
%
\ifx \collEname  \itsempty
\relax % 如果沒有學院名 (英文),則不佔版面,不印任何空白
\else
% 有學院名 (英文)
\large{\collEname}\\% 學院名 (英文)
\fi
%
\large{\univEname}\\  %英文校名
\vspace{0.83cm}
%\vfill
%
\large{ABSTRACT}\\
%\vspace{0.5cm}
\end{center}
% Resume the line spacing the desired setting
\renewcommand{\baselinestretch}{\mybaselinestretch}   %恢復原設定
%\large %it needs a font size changing command to be effective
% restore the font size to normal
\normalsize
%%%%%%%%%%%%%
%Although the global economy is facing recession worries, the sales strength of smart phone did not see slow down. The global handset shipments reach 1.6 billion sets in 2011, annual growth rate of only 11\% (excluding white box cell phone) the smart phone shipments of 450 million, a growth rate of more than 60\%, penetration rate of 27.95\%, the Topology Research Institute. 
      
%Because the driven both of emerging market demand and the parity trend commodities, expected 2012 growth momentum continued in smart mobile phone, the shipments approach 600 hundred million mark, the penetration rate of more than one-third predicted that by 2015, half of the world mobile phones are for the world of smart phones. 

%However when smart phones has NFC function, except to grasp whether the existing use of smart phones life would more convenient or not, to pay attention to the degree of the user are expected, or are still under observation of impact. 

%In this study, the Technology Acceptance Model (TAM), perceived usefulness and perceived ease of use questionnaire architectural foundation for the two influencing factors, coupled with the demand for mobile phones (dependence), the acceptance of new technology (satisfaction) and an additional fee on the use of NFC applications to an acceptable level of the three dimensions of variables. To understand the extent of changes in consumer behavior to use the Smartphone’s NFC function, and the SPSS 19.0 statistical analysis software to conduct analysis to deal with hypothesis testing. 

%The results showed that the perceived usefulness and perceived ease of use, level of demand for mobile phones (dependence), acceptance of new technology acceptance (satisfaction) and the use of NFC applications at an additional cost, all will positive impact on consumer behavioral intention to use the NFC smart phones.

%Keywords: mobile communication, NFC (Near field communication), smart phones, the technology acceptance model (TAM), theory of consumer behavior.


%%%%%%%%%%%%%%%%%%%%%%%%%%%%%%%
%       誌謝
%%%%%%%%%%%%%%%%%%%%%%%%%%%%%%%
%
% Acknowledgment
\newpage
\chapter*{\protect\makebox[5cm][s]{\nameAckn}} %\makebox{} is fragile; need protect
\phantomsection % for hyperref to register this
\addcontentsline{toc}{chapter}{\nameAckn}
本篇論文能如期完




%%%%%%%%%%%%%%%%%%%%%%%%%%%%%%%
%       目錄
%%%%%%%%%%%%%%%%%%%%%%%%%%%%%%%
%
% Table of contents
\newpage
\renewcommand{\contentsname}{\protect\makebox[5cm][s]{\nameToc}}
%\makebox{} is fragile; need protect
\phantomsection % for hyperref to register this
\addcontentsline{toc}{chapter}{\nameToc}
\tableofcontents

%%%%%%%%%%%%%%%%%%%%%%%%%%%%%%%
%       表目錄
%%%%%%%%%%%%%%%%%%%%%%%%%%%%%%%
%
% List of Tables
\newpage
\renewcommand{\listtablename}{\protect\makebox[5cm][s]{\nameLot}}
%\makebox{} is fragile; need protect
\phantomsection % for hyperref to register this
\addcontentsline{toc}{chapter}{\nameLot}
\listoftables

%%%%%%%%%%%%%%%%%%%%%%%%%%%%%%%
%       圖目錄
%%%%%%%%%%%%%%%%%%%%%%%%%%%%%%%
%
% List of Figures
\newpage
\renewcommand{\listfigurename}{\protect\makebox[5cm][s]{\nameTof}}
%\makebox{} is fragile; need protect
\phantomsection % for hyperref to register this
\addcontentsline{toc}{chapter}{\nameTof}
\listoffigures

%%%%%%%%%%%%%%%%%%%%%%%%%%%%%%%
%       符號說明
%%%%%%%%%%%%%%%%%%%%%%%%%%%%%%%
%
% Symbol list
% define new environment, based on standard description environment
% adapted from p.60~64, <<The LaTeX Companion>>, 1994, ISBN 0-201-54199-8
% 目前用不到
%\newcommand{\SymEntryLabel}[1]%
%  {\makebox[3cm][l]{#1}}
%%
%\newenvironment{SymEntry}
%   {\begin{list}{}%
%       {\renewcommand{\makelabel}{\SymEntryLabel}%
%        \setlength{\labelwidth}{3cm}%
%        \setlength{\leftmargin}{\labelwidth}%
%        }%
%   }%
%   {\end{list}}
%%%
%\newpage
%\chapter*{\protect\makebox[5cm][s]{\nameSlist}} %\makebox{} is fragile; need protect
%\phantomsection % for hyperref to register this
%\addcontentsline{toc}{chapter}{\nameSlist}
%\input{my_symbols.tex}


\newpage
%% 論文本體頁碼回復為阿拉伯數字計頁,並從頭起算
\pagenumbering{arabic}
%%%%%%%%%%%%%%%%%%%%%%%%%%%%%%%%

% generates the cover page
% 不使用預設的 title 
%\maketitle

%\fontsize{14}{20}\selectfont
%\setlength{\baselineskip}{24pt}
% insert the table of contents
%\tableofcontents
%\listoffigures
%\listoftables

\chapter{緒論}

\section{研究背景與動機}
我們的生活因為科技的蒸蒸日上的不斷進步的年代,在發展迅速的資訊社會中,社會不斷演變出現各式各樣的網路工具與平台,例如:即時通訊軟體、留言板、論壇、 社交網站、 部落格、 網誌、購物網等等,都成為了現在社會人們收集資訊的工具,比起以只能在實體店面取得所需資訊或購買到所要的商品,如今人們改變了傳統的購物方式,許多人們改用方便的網路通路,許多廠商紛紛觸角伸往網路商店虛擬通路的市場,網路商店這一個新興的市場,網際網路多樣化的平台的優點如:不受空間與時間所影響還可以減少人力成本降低企業的經營成本,以及透過近年來流行的各大工具與平台來行銷,提升品牌知名度相當有效,例如經營fackbook粉絲專業,社群網路粉絲專頁社群媒體品牌行消策略研究,等可有效提升知名度等。

隨著web2.0與web3.0的不斷成熟在各大平台與工具的強大的幫助下社群媒體品牌行銷也逐漸變成廠商紛紛投入虛擬網路通路的一環,因此本研究以La Jolla 樂活雅鈦鍺精品的例子來研究與探討,網路品牌聲望與網路品牌知覺是否影響消費者行為。

根據資策會2010年調查結果顯示\cite{資策會FIND},台灣家庭中上網普及率為82.8%,與2009年比較,微幅上升4.1個百分點如圖~\ref{fig:yuwei_2011020901}所示,估計近期間653萬家戶有上網,較去年增加44萬戶上網家庭。從近幾年家庭上網比例來看,台灣家庭上網率在去年微幅成長後,今年呈現顯著成長趨勢,家庭上網率亦突破80.0%大關。

\begin{figure}[htbp]
\centering \includegraphics[%
  width=13cm,keepaspectratio]{images/yuwei_2011020901}
\caption{\label{fig:yuwei_2011020901}台灣家戶連網普及率}
%(資料來源:資策會FIND)
\end{figure}

從調查數據如圖~\ref{fig:yuwei_2011022005}所示可以發現兩個趨勢:首先,上網民眾的網路活動更加活躍,在上傳與下載檔案、從事線上影音等活動,都有相當顯著的成長,其次,民眾使用網路交易的比例倍增,顯示民眾對於網路的虛擬購物環境,已有相當程度的信任。\cite{資策會FIND}


\begin{figure}[htbp]
\centering \includegraphics[%
  width=13cm,keepaspectratio]{images/yuwei_2011122005}
\caption{\label{fig:yuwei_2011022005}2011/12月曾上網者之連網使用行為}
%(資料來源:資策會FIND)
\end{figure}

根據以上資料可以了解到近年來網路發展迅速因此消費者習慣也開始改變紛紛改用網路購物因此所謂的『宅經濟』的興起。宅經濟又稱為『閒人經濟』是指不用出門在家就可以從事經濟活動。
根據圖~\ref{fig:Onlinetrends}臺灣線上購物的市場規模自 2006 年 開始到現今每年 二位數的成長趨勢,2010 年市場規模為 3,583 億元,比 2009 年成長了 15%;預估 2011 年線上購物的市場規模可達到 4,300 億元\cite{資策會FIND}。他是個成長非常迅速的市場。

\begin{figure}[htbp]
\centering \includegraphics[%
  width=13cm,keepaspectratio]{images/Onlinetrends}
\caption{\label{fig:Onlinetrends}資策會 MIC,2010 台灣線上購物市場規模 3,583 億元,(2012 年 6 月),}
%(資料來源:資策會FIND)
\end{figure}

在近年來網路社交興起,公司紛紛採取社群網站來提升公司的知名度或來作公司產品行銷因此近年產生出新的詞『社群媒體行銷』,觀察網友在最近很火紅的Facebook使用行為上,發現網友使用Facebook的平均使用時間高達439.5分鐘,平均下來,每天約黏在Facebook上面14.65分鐘,已佔了使用社群網站時間的56.6%(資策會FIND)根據互動行銷機構 Rosetta 調查顯示,全球百大零售商已經有 59%在 Facebook 擁有官方粉絲專頁簡稱粉絲團(羅之盈,2010),由此可見,企業也以觀察到Facebook也是一個強大的行銷手法工具之一,透過粉絲專業可以發展出全新的消費者市場,透過Facebook互通性來迅速散播公司相關資訊與雙向溝通可把從粉絲專頁中的人導入公司網路消費者逐群。

但是因為網路通路與實體通路不同的地方是,網路通路所販售的所有商品並不像實體通路有專人解說介紹商品,網路上的消費者只能透過網頁瀏覽器(Web)來觀看網頁上所想購買的商品並沒辦法隨時有像實體通路有專人銷售員介紹的每項功能或使用方法,企業也無法有效掌握顧客的所有需求與企業消費者對商品的滿意度,所以更難推測出未來顧客的消費者行為。



\section{研究目的}
本研究目的是在瞭解透過網路多媒體行銷後的廠商,在網路品牌聲望與品牌知覺是否會影響消費者行為進行分析與推測,提供企業在網路多媒體行銷方面提供有效的建議可以透過本研究更瞭解網路消費者所需求與提升未來企業經營網路虛擬通路之參考


\section{研究流程}

 本研究之流程如圖所示,步驟如下:

1.確定研究動機與範圍

         本研究目的主要目的要探討品牌知覺、品牌聲望、與網路消費者購買行為之關係,研究範圍界定於台灣地區的網路消費者在網路購物中於鈦鍺時尚精品之交易行為。

2-1.文獻收集與研讀

        瞭解本研究的界定主範圍後,開始收集品牌知覺、品牌聲望、網路消費者購買行為、鈦鍺時尚精品等相關文獻,作為研究的基本理論。

2-2. 進行實地訪談

        為了瞭解本研究的鈦鍺時尚精品,相關公司所遇到的問題與消費者行為,前往La Jolla 樂活雅 鈦鍺精品公司實地訪談。

3-1.發展問卷架構

       本研究經過文獻資料與整理及研究探討之後,根據文獻資料與本研究的方向,建立其問卷研發與架構及研究變數,再來針對各個本研究的變數建立研究假設與操作性定義。

3-2.客群分析與訪查

      經過實地親自探訪收集La Jolla 樂活雅 鈦鍺精品公司,所得知的相關客群資料後開始客群分析與訪查本研究相關資料。

4.問卷制作與發放

      本研究方式是採用發放問卷的研究方式,對研究對象進行相關資料的調查;根據研究資料與本研究主題下去研究設計製做問卷,依據本研究該需要的變數去設計選題,即可以開始發放本研究正式的調查問卷,其問卷為網際網路發放方式為收集為主,研究流程如圖~\ref{fig:NPC12}所示。

5.問卷整理與數據分析

      首先針對本研究收回的問卷樣本作敘述性統計分析,了解問卷樣本的基本特性。

6.結論與探討

     依據本研究分析後研究出的結果,統整成文後作研究結論與探討,以作為後續與本文相關研究學者與人員參考之文獻。

\begin{figure}[h]
\centering \includegraphics[%
  width=12cm,keepaspectratio]{images/NPC12}
\caption{\label{fig:NPC12}研究流程}
%(資料來源:本研究整理)
\end{figure}



%\section{研究架構}


%\chapter{文獻探討}
%\section{NFC}
%\section{理性行為理論}
%\section{科技接受模式}
%\section{便利性}
%\section{系統品質}
%\section{認知安全性}
%\section{相容性}
%\section{社會影響力}
%
%\chapter{研究設計與方法}
%\section{研究架構}
%\section{研究假說}
%\section{研究構面之操作型定義與衡量問項}
%\section{研究對象}
%\section{資料分析}
%
%\chapter{資料分析與實證研究}
%
%\section{敘述統計分析}
%\section{信度分析}
%\section{建構效度}
%\section{相關分析}
%\section{迴歸分析}
%
%\chapter{研究結論與建議}
%
%\section{研究結論}
%\section{研究貢獻}
%\section{後續研究之建議}
%

\chapter{文獻探討}


\section{品牌的定義}

根據各學者指出品牌的定義如以:

美國行銷學會(American Marketing Association,AMA)
對品牌的定義為:名稱 (Name)詞語(Term) 標記(Sign)象徵(Symbol )設計(design)\cite{AMA}

Sappington and Wernerfelt (1985) \cite{Sappington}品牌名稱是企業的一項貴重資產,可以提升消費者對於產品的需求性,減少顧客的不確定感,且品牌名稱也被當成是對公司產品的品質保證。

Farquhar (1989) \cite{Farquhar}品牌是一個名稱、符號、設計或標誌,可以使一個產品增加功能利益還有以外的價值。

Rao and Ruekert (1994) \cite{Rao}品牌對於消費者而言是屬於產品的一部分,可視為傳遞產品品質訊息的媒介,而且往往是消費者購買決策的重要考量因素之一,具有增加產品價值的功能,而且可以提供產品一定的品質保證,降低消費者的搜尋成本及知覺風險。

Boyd, Walker, 和 Larreche (1995)\cite{Boyd} 認為品牌的構成,成分一般可以細分為:
\begin{enumerate}
\item品名(brand name):可以發聲唸出的部分。
\item品牌標誌(brand mark):不能以言語表達的部分,例如符號、設計或獨特的包裝。 
\item商標(trademark):法律上,專屬於某一個賣方的品牌或品牌的某部分。
\end{enumerate}

另外,根據 Laforet 和 Saunders (1994)\cite{Laforet} 的實證研究,歸納出三種品牌導向,六種品牌策略,這三種品牌導向所引發出的六個策略為:
\begin{enumerate}
\item 企業品牌導向

1.企業品牌策略:直接採用企業名稱作為產品品牌名稱。 

2.部門品牌策略:當企業跨足不同市場,單一企業名稱不敷使用時,不同部門採用不同附屬名稱 (subsidiary name) 作為產品品牌名稱
\item 產品品牌導向

1.個別品牌策略:各產品專屬品牌,不刻意突顯企業名稱,但在包裝上某處仍會出現。

2.獨立品牌策略:各產品專屬品牌,但是刻意不揭露企業名稱,不希望消費者知道是哪一家企業製造。
\item 混合品牌導向

1.雙重品牌策略:兩個或更多層級的品牌要素,組合而成產品品牌,而且每個品牌要素的顯著性相等。

2.背書式品牌策略:同樣以兩種品牌要素組成品牌,但企業或家族品牌作為背書之用,新產品的品牌較為顯著。
\end{enumerate}

\begin{table}[htb]
\caption{文獻整理}
\label{tab:PL1}
\centering
%\renewcommand{\arraystretch}{1.2} % 將表格行間距加大為原來的 1.2 倍
%\arrayrulewidth=1pt % 調整線條粗細為 1pt
%\tabcolsep=24pt % 調整欄間距為 24pt
%\begin{document}
\begin{tabular}[t]{|c|p{8.5cm}|p{2.5cm}|} % 第一欄位使用 sans serif 字族
\hline
學者&定義 & 年代 \tabularnewline
\hline
美國行銷學會 & 對品牌的定義為:名稱 (Name)詞語(Term) 標記(Sign)象徵(Symbol )設計(design)&  \tabularnewline
\hline
Sappington and Wernerfelt  &企業的品牌名稱是一個寶貴的資產,可以提升消費者對
於產品的需求性,減少顧客的不確定感,且品牌名稱也被當成是對公司產品的品質保證。& 1985 (施瑩婕(2009)譯) \tabularnewline
\hline
Farquhar &品牌是一個名稱、符號、設計或標誌,可以使一個產品
增加的不僅是功能利益還有功能利益以外的價值。&1989 (施瑩婕(2009)譯) \tabularnewline
\hline
Rao and Ruekert &品牌對於消費者而言是屬於產品的一部分,可視為傳遞產品品質訊息的媒介,而且往往是消費者購買決策的重要考量因素之一,具有增加產品價值的功能,而且可以提供產品一定的品質保證,降低消費者的搜尋成本及知覺風險。&1989(施瑩婕(2009)譯)  \tabularnewline
\hline
 Laforet 和 Saunders&歸納出三種品牌導 :(一)企業品牌導向(二)產品品牌導向(三)混合品牌導向
&1994\tabularnewline
\hline
oyd, Walker, 和 Larreche&認為品牌的構成,成分一般可以細分為:(一)品名(brand name)(二)品牌標誌(brand mark)(三)商標(trademark)&1995\tabularnewline
\hline
\end{tabular}
\end{table}

以上學者

\section{品牌知覺}
Aaker (1991) \cite{Aaker1991}將消費者對品牌的知覺品質定義為消費者對於某一 項品牌產品整體品質的認為水準,或消費者對在特定目的下相對於其他品牌, 對某品牌產品或服務全面品質的主觀滿意程度。

kelle(1993)\cite{Keller1993}認為品牌認知是指在消費者記憶中較強的品牌聯想與連接。
將品牌知覺的衡量主要分成三種,敘述如下:
\begin{enumerate}
\item 屬性:消費者在消費之餘所認知的品牌是什麼、品牌有什麼。
\item 利益:指消費者個人價值。
\item 態度:指消費者對品牌整體的評價
\end{enumerate}

%Dawar and Parker(1994) 指出消費者挑選產品以品牌聲望為主要考量
%Herbig and Milewica(1996) 正向的品牌聲望有助收益增加。
%Kowallczyk and Pawlish(2002)公司聲望是影響消費者購買的因屬之一。
%Veloutsou and Moutinho(2009)維持和提升公司聲望比強化消費者滿意度更重要
%Priporas and Kamenidou(2011)品牌聲望佈警示品牌保證,也是行銷推廣的利器

keller (1998)\cite{Keller1998}提出品牌的意義與功能可以從四種角度來說明:
\begin{enumerate}
\item 品牌可以用圖案來辨別,可用來與競爭者來區別

\item 品牌一致的保證與承諾,是消費者在購買或使用之前的感覺產品的價值與品質

\item 品牌是可以自我投射形象,品牌個性的傳達

\item 品牌不僅是一組有關產品的定位,代表一致性品質與功能性的集合,也可作為消費者決策購買時的線索
\end{enumerate}
    Keller (1998)\cite{Keller1998}亦詳細地指出,品牌權益實包含:
\begin{enumerate}
\item 品牌鮮明度 (Brand Salience):可能會影響消費的判別難易度
\item 品牌績效 (Brand Performance):可以滿足消費者所需功能
\item 品牌形象 (Brand Image):在消費者心中產生對品牌抽象整體概念
\item 品牌判斷 (Brand Judgment) :消費者對於品牌理性層面的判定
\item 品牌情感 (Brand Feeling):消費者對品牌情感的概念與特性,或是社會認可的特徵
\item 品牌共鳴 (Brand Resonance):與消費者品牌關係的最高層次,是由品牌情感到具體行動購買的具體表現,例如主動參與及重複購買的行為忠誠度。
\end{enumerate}
Aaker (1996)\cite{Aaker1996}認為品牌權益可分為:(1)品牌知名度 (Brand Awareness)、(2)品牌忠誠度 (Brand loyalty) 、(3)品牌知覺品質 (Brand perceived quality) 、(4)品牌聯想 (Brand Association)、(5)其他品牌專屬資產 (Brand Speciality Asset) 。


%\begin{figure}[htbp]
%\centering \includegraphics[%
 % width=13cm,keepaspectratio]{images/keller2008}
%\caption{\label{fig:keller}來自keller2008}
%(資料來源:本研究整理)
%\end{figure}



\begin{table}[htb]
\caption{文獻整理}
\label{tab:PL2}
\centering
%\renewcommand{\arraystretch}{1.2} % 將表格行間距加大為原來的 1.2 倍
%\arrayrulewidth=1pt % 調整線條粗細為 1pt
%\tabcolsep=24pt % 調整欄間距為 24pt
%\begin{document}
\begin{tabular}[t]{|c|p{8.5cm}|p{2.5cm}|} % 第一欄位使用 sans serif 字族
\hline
學者&定義 & 年代 \tabularnewline
\hline
Aaker & 將消費者對品牌的知覺品質定義為消費者對於某一 項品牌產品整體品質的認為水準,或消費者對在特定目的下相對於其他品牌, 對某品牌產品或服務全面品質的主觀滿意程度。& (1991)  \tabularnewline
\hline
kelle&將品牌知覺的衡量主要分成三種,敘述如下:屬性,利益,態度& 1993 \tabularnewline
\hline
Aaker&認為品牌權益可分為:(1)品牌知名度 (Brand Awareness)、(2)品牌忠誠度 (Brand loyalty) 、(3)品牌知覺品質 (Brand perceived quality) 、(4)品牌聯想 (Brand Association)、(5)其他品牌專屬資產 (Brand Speciality Asset) &1996。\tabularnewline
\hline
Keller&提出品牌的意義與功能可以從四種角度來說明:品牌可以用圖案來辨別,品牌一致的保證與承諾,品牌是可以自我投射形象,品牌不僅是一組有關產品的定位。&1998 \tabularnewline
\hline
Keller&指出品牌權益實包含:品牌鮮明度 (Brand Salience),品牌績效 (Brand Performance),品牌形象 (Brand Image),品牌判斷 (Brand Judgment),品牌情感 (Brand Feeling),品牌共鳴 (Brand Resonance),例如主動參與及重複購買的行為忠誠度。&1998 \tabularnewline
\hline
\end{tabular}
\end{table}

以上學者對於「品牌」的定義可以瞭解,許多學者舉出的品牌定義並不一致,但可以瞭解出品牌是公司無形有力的資產,由以上可看出品牌對公司來說非常的重要不但能提升公司的營運與消費者對公司的認知

\section{消費者行為}
早期的消費者行為通常都以消費者動機來當研究核心,隨著各位學者長年研究下來提出了許多相關理論的模式,但是現在研究後期主要都已決策的過程為主要核心

Pratt(1974)\cite{Pratt1974}消費者行為是指購買行動,其中購買的行動至 少包括四個因素:
\begin{enumerate}
\item 購買主體-購買者本身; 
\item  購買物品或勞務;
\item  購買媒介-現金或支 付的承諾;
\item  決定購買的行動。
\end{enumerate}

Walter&Gordon(1970)\cite{Walter1970})指人們在購買、使用產品或服務時的相關行為。另一位學者林靈宏(2000)\cite{林靈宏}將消費者個體視為一個心理單位,消費者的價 值觀、知覺、學習、人格、動機、經驗、記憶、 認知、態度及涉入程度都會影響消費行為。

Schiffman&Kanuk(2003)\cite{Schiffman}瞭解消費者是如何進行決策,以支配可得資源 (時間、金錢、努力)於各種消費項目。這些決策 包括購買何種物品 (what)?為何而買 (why)? 何時購買 (when)?在哪裡購買 (where)?多常 購買 (how often)?以及使用 (use) 頻率。

\section{消費者行為理論(Consumer Behavior Theory)}
\subsection{EBK消費者行為定義}
消費者行為可以定義於:消費者在搜尋、評估、購買、使用和處理一項產品、服務、和理念(ideas)時表現的行為。\cite{EBK}
\begin{figure}[htbp]
\centering \includegraphics[%
  width=16cm,keepaspectratio]{images/ConsumerBehaviorTheory}
\caption{\label{fig:ConsumerBehaviorTheory}Engel, Blackwell and Kollat(1993) 汪澤普(2007)整理}
%(資料來源:本研究整理)
\end{figure}

消費者決策行為模式有好幾種,其中較為完整且具系統性的模式架構的
是EKB Model(Engle, Kollat & Blackwell Model)。
Engel, Blackwell, and Kollat 等學者在 1968 年從消費者行為理論中,發展出了 EKB 模型 (Engel-Kollat-Blackwell Model, EKB Model),
在 EKB 模型中,可以分為四大部分:
\begin{enumerate}
\item輸入(Input):消費者所接受的外界訊息,其主要來自於兩方面:一是非行銷來源,如大眾傳播媒體或人際溝通管道;二是從行銷來源,如廠商的行銷活動。
\item資訊處理(Information Processing):資訊處理是經由刺激的接受、中斷與記
憶的儲存和稍後取用的過程,可分為展露(Exposure)、注意(Attention)、理
解(Comprehension)、接受(Acceptance)及保留(Retention)等五個步驟。
\item決策過程(Decision Process):決策過程可分為五個階段,是EKB模型中最 主要的部分,
此五個階段雖然是線性的過程,但卻可以反覆地回到之前的階段(Zellwegger 1997),此五個階段敘述如下:

a. 需求確認(Problem Recognition):是消費者決策過程中的第一個階段,當消費者認知到現實與理想狀態存在著差距時,就會意識到需求的存 在,而這些需求有可能會被外部的或內部的因素所觸發,比如:廠商的 促銷活動或者是個人的經濟狀況提升等等。

b. 資訊搜尋(Information Search):消費者在確認了需求動機後,就會開 始搜尋資訊,搜尋的範圍包括了記憶中的知識及外部的環境,前者稱為 內部搜尋,後者則稱為外部搜尋。

c. 選擇評估(Alternative Evaluation):當消費者取得了足夠的資訊後,即 會對可能的選擇方案加以分析與評估,來做為後續制訂購買決策的依 據。

d. 購買(Purchase):經過審慎的分析與評估後,消費者會從選擇方案中選 擇其一購買;消費者在這階段中,也必須決定從何處以及如何購買。

e. 購後行為(Post-purchase):消費者在使用或消費所選擇的商品或服務 後,會對其做出評估,以做為下此購買的參考。

\item影響決策變數(Variables Influence Decision Process):影響決策過程的變數可 分為兩部分:一為環境因素,包括文化、社會階層、個人影響、家庭與情境 等因素;二為個人差異,包括消費者資源、動機與涉入、知識、態度、人格、 價值觀與生活形態。
\end{enumerate}



\begin{table}[htb]
\caption{文獻整理}
\label{tab:PL3}
\centering
%\renewcommand{\arraystretch}{1.2} % 將表格行間距加大為原來的 1.2 倍
%\arrayrulewidth=1pt % 調整線條粗細為 1pt
%\tabcolsep=24pt % 調整欄間距為 24pt
%\begin{document}
\begin{tabular}[t]{|c|p{8.5cm}|p{2.5cm}|} % 第一欄位使用 sans serif 字族
\hline
學者&定義 & 年代 \tabularnewline
\hline
Pratt1974 &消費者行為是指購買行動,其中購買的行動至 少包括四個因素:(1)買主體-購買者本身; (2)購買物品或勞務;(3)購買媒介-現金或支 付的承諾;(4)決定購買的行動。
& 1974  \tabularnewline
\hline
Walter&指人們在購買、使用產品或服務時的相關行為。& 1970 \tabularnewline
\hline
林靈宏&將消費者個體視為一個心理單位,消費者的價 值觀、知覺、學習、人格、動機、經驗、記憶、 認知、態度及涉入程度都會影響消費行為 &2000。\tabularnewline
\hline
Schiffman&Kanuk&瞭解消費者是如何進行決策,以支配可得資源 (時間、金錢、努力)於各種消費項目。這些決策包括購買何種物品 (what)?為何而買 (why)? 何時購買 (when)?在哪裡購買 (where)?多常 購買 (how often)?以及使用 (use) 頻率。&2003 \tabularnewline
\hline
Engel, Blackwell, and Kollat &EKB 模型中,可以分為四大部分:輸入(Input),資訊處理(Information Processing),決策過程(Decision Process),影響決策變數(Variables Influence Decision Process)&1968  \tabularnewline
\hline
\end{tabular}
\end{table}

\subsection{影響消費行為的心理因素}
消費者行為可以定義於:消費者在搜尋、評估、購買、使用和處理一項產品、服務、和理念(ideas)時表現的行為。\cite{林靈宏}
\begin{figure}[htbp]
\centering \includegraphics[%
 width=13cm,keepaspectratio]{images/影響消費行為的心理因素}
\caption{\label{fig:影響消費行為的心理因素}林靈宏 (2000),消費者行為學,台北:五南圖書出版公司。}
%(資料來源:本研究整理)
\end{figure}


\subsection{消費者決策行為在傳統市場與網路市場中的差異}

\begin{figure}[htbp]
\centering \includegraphics[%
  width=16cm,keepaspectratio]{images/Difference}
\caption{\label{fig:ConsumerBehaviorTheory}Bulter and Peppard(1998) 汪澤普(2007)整理}
%(資料來源:本研究整理)
\end{figure}


\section{影響消費購買意願}
\subsection{購買意願之購買決策過程}
Engel, et al (2001) 認為,消費者決策過程的五個重要階段如圖~\ref{fig:Engel}
\begin{enumerate}
\item 問題認知

購物過程開始時於消費者觀察自身需求與問題來源的認知 ;消費者在問題認知方面會受到外在因素影響(文化,人口統計變數,慘考群體)與個人因刺激,引起消費者產生動機

\item 尋求

消費者在確定自己本身所需後,會根據自身所需或是問題來尋求相關資訊,以進行購物決策。一般的消費者收集資訊通常都分為兩種來源,內部搜尋與外部搜尋;消費者,通常會先從自身的記憶中搜尋所需的相關資訊,如過記憶中沒有相關記憶就會改以外部搜尋,獲得協助決策的相關資訊,外部搜尋的資訊如:家人,朋友,廣告,網路等。
           
\item 方案評估

搜尋資料完後,就可以對所需要的選著方案做評估與最後的決策。通常消費者都是透過各項評估的標準與尺度來評定購買的方案。

\item 選擇

經過以上敘述方案評估過程後,消費者會以全部方案中選擇最適合的方案,並購買的行動。

\item 結果 

當消費者購買產品後,因為本身對產品的期望結果與實際使用的結果,兩種之間的感覺受差異。
一般消費者購買的商品使用後心裡的感受主要有三中結果

符合期望:消費者使用購買的產品後的結果表現符合預期的期望,沒有特別好或壞的感覺

非常滿意:消費者使用購買的產品後的結果表現超過預期的期望,導致心裡的感覺很滿意

不滿意:消費者使用購買的產品後的結果表現低於預期的期望,導致心裡的感覺很不滿意的反應
\end{enumerate}

\subsection{消費者滿意度}
消費者滿意度:

     「消費者滿意度」最早由 Cardozo 於 1964 年所提出以實證研究方式探討顧客預期與實際感受之差 距對滿意度以及滿意度對再購意願之影響。

Fornell(1992)指出消費者滿意度是指顧客在購買產品 或使用服務後主觀性的整體衡量。


Oliver (1977, 1980,1981) \cite{Oliver1980}視為滿意度的基礎架構,消費者在購買前會先對產品或服務產生期望,與對產品或服務購後使用的實際績效表現相互比較,產生滿意程度上的差異,當所感受的實際績效高於期望時,稱為正向不確認, 亦即當所感受的實際績效低於期望時,稱為負向不確認,當正向不確認越高即滿意度越 高時,消費者將會持續購買或使用產品或服務,當負向不確認越高時,消費者將不購買 或持續使用商品或服務顧客滿意是一種特定交易的反應。 一種購買的結果,指消費者比較購。

Turner(2011)則提到,顧客滿意度包含了幾項影響因子,分別為產品(特色、品質、獨特 性 )、與可替代品的價格比較 、 競爭環境、品牌與其產品之間的情感聯繫、與顧客以往的互動所建立的商譽及其他內外在因素(醜聞、政經環境等)。總而言之,消費者滿意度是結合消費者認知及情感的綜合評量結果。

陳思璇(2011)\cite{陳思璇}提出 資訊品質及服務品質顯著地影響重購顧客之滿意度,且顧客滿意顯著地影響重購顧客之再購買意願。然而,系統品質對重購顧客而言效果並不顯著。另位張倫嚴(2012)\cite{張倫嚴}另位產品創新、產品涉入、知覺價值與顧客滿意度間具有顯著正向影響,

藍悅真(2012)\cite{藍悅真}提出1.品牌形象的「功能性」、「經驗性」能正向預測消費者滿意度的產品滿意。2.品牌形象的「功能性」、「象徵性」能正向預測消費者滿意度的服務滿意。3.品牌形象的「功能性」、「經驗性」能正向預測消費者滿意度的整體滿意。4.品牌形象的「功能性」、「經驗性」能正向預測消費者忠誠度。本研究除了根據分析結果予以解釋說明外,亦對研究結果的應用加以分析討論,並對後續研究提出一些建議。

以上學者對於「消費者滿意度」的定義可以瞭解,雖然個學者們沒有一致的定義,但都有都提到消費者滿意度度乃建構在購買前消費者對商品與服務的預期與購買獲得的效益認知差異,兩者差距用大時顧客對於滿意與不滿意越明顯。



\begin{figure}[htbp]
\centering \includegraphics[%
  width=13cm,keepaspectratio]{images/Engel}
\caption{\label{fig:Engel}Engel, et al (2001)}
%(資料來源:本研究整理)
\end{figure}


\section{La jolla樂活雅}
\begin{enumerate}
 \item 經營據點:台灣:台北市大安區光復南路446號七樓 
               美國:2231 N. 23rd St. Beaumont, TX 77706

 \item  經營理念:
富紳國際實業有限公司主要從事首飾及貴金屬零售業;擁有為數不少的客戶群。
最初,是由兩位業餘華裔設計師,自行創作手工珠寶藝品,由於設計風格獨特,大受消費者的歡迎及喜愛,總是供不應求。設計師們有鑑於手工飾品無法大量生產,又觀察到,在未來的飾品趨勢中,鈦飾品將取代金銀等貴重金屬,成為高級珠寶設計重要材質,因此決定發展鈦鍺飾品。

以生產設計鈦鍺精品為主的富紳國際,也接受客戶OEM及ODM訂單。舉凡項鍊、手鍊、戒指、袖扣等商品,並供貨給日本、美國等客戶及珠寶精品業者。
另外,富紳國際亦提供客製化服務,承接結合鈦飾品及頂級鑽石之客製化訂單,為消費者
提供獨一無二的商品訂製服務。

發展至今,富紳國際已擁有自己的設計師及協力生產工廠,從來圖打樣、來樣製作及專業生產一應俱全。專屬設計師群發揮豐富的創造力,賦與每一款設計精品獨創的理念,再結合老師傅精湛的珠寶鑲工技藝,精心打造每一分一毫細微處,提供國內外客戶最與眾不同、匠心獨具的頂級鈦鍺珠寶精品。

 \item 企業文化:La Jolla品牌概念來自於美國加州。 La Jolla,源於西班牙文「珠寶」之意-「聖地牙哥的海洋之珠」,因為是西班牙文,所以唸法獨特,J的發音為H的氣音,唸作[ la-ho-ya](接近中文發音”拉荷亞)。La Jolla是一個位於美國加州San Diego的明媚小鎮。

在「陽光.沙灘.美麗海岸」的見證下,品牌創辦人遇見了”命中注定”的另一半,並於 La Jolla
小鎮買下了別具意義的定情戒,兩人互許真愛,相守一生。這份難能可貴的愛情,讓他們
決心將這份浪漫,轉化為璀燦迷人的健康概念純鈦飾品,將他們勇於追尋真愛的故事,透
過La Jolla精品,不斷的傳遞出去。

La Jolla品牌理念-Pure and  Trendy

Pure 材質純度嚴選 

Trendy 引領現代時尚精品風格 

「 La Jolla期許能結合藝術、自然、愜意美好的一切事物,以獨到細膩的品味設計、純鈦
的材質,帶來對生命最美好的感動。」 有別於一般市售的大眾化純鈦飾品, La Jolla
品牌創辦人堅持自我品味,精心打造一個具有現代時尚個性的純鈦精品。旗下專業設計師
更以獨創的設計理念,賦與每款飾品生命力,並結合老師傅細緻的鑲工手藝,極致講究每
一分每一毫細微的作工,呈現出純鈦精品的高雅質感及活力,演繹品味獨具的 La Jolla
純鈦精品新魅力。
 
\item 品牌故事

有別於一般市售的大眾化鈦鍺飾品,La Jolla品牌創辦人CORA LIU堅持自我品味,精心打造一個具有現代時尚個性的鈦鍺精品。旗下專業設計師更以獨創的設計理念,賦與每款飾品生命力,並結合老師傅細緻的鑲工手藝,極致講究每一分每一毫細微的作工,呈現出鈦鍺飾品的高雅質感及活力,演繹品味獨具的La Jolla鈦鍺精品新魅力。
\end{enumerate}

\begin{figure}[htbp]
\centering \includegraphics[%
  width=16cm,keepaspectratio]{images/LaJolla}
\caption{\label{fig:LaJolla}LaJolla樂活雅鈦鍺精品}
%(資料來源:本研究整理)
\end{figure}



\chapter{研究過程}
寫出你的研究過程與相關資料收集的內容.....

\blindtext[1]


\chapter{研究結果展示}
%\section{研究分析}
資料分析方法
本研究利用統計分析套件軟體 SPSS 20進行相關分析,使用的分析如下
\begin{enumerate}
\item 描述性統計 :分析樣本結構中性別,年紀教育,程度工作性質的百分比
\item 信度分析:以Cronbach's 值來鑑定各量表的內部一致性
\item 因素分析:主要目的將原有很多變數(維度)之資料,縮減成較少的維度數,但有保持原本所提供的資料
\item 簡單迴歸:分析進行假設的驗證
\end{enumerate}

本研究探討品牌知覺、品牌聲望、消費者滿意度之間的關係。本研究已發放問卷的方式取的資料,問卷填寫主要以網路發放題目主要
品牌知覺、品牌聲望、消費者滿意度以簡單迴歸分析來驗證假設是否成立

變數的定義與衡量
\begin{enumerate}
\item  消費者行為共有10個題目在網路上選購精品、珠寶商品時會參考的因素:價錢、品牌形象、製造材料與特性、其他顧客反應度、商品認證保證、售後服務、交貨服務、整體商品的價值、品牌知名度、設計感
\item 品牌知覺共有8個題目知覺對鈦鍺的所產生出來的狀況:商品的品質良好、鈦鍺品牌形象、製作材料(與特性)、商品外觀設計很滿意、商品的廣告或名稱很滿意、整體商品的價值、品牌知名度
\item 品牌聲望共有10個題目得到相關資訊是否有幫助您了解LaJolla鈦鍺精品:價錢、品牌形象、製作材料(與特性)、其他顧客反應、商品認證保證、售後服務、交貨服務、整體商品的價值、品牌知名度、設計感
\item 消費者滿意度共有10題目購買相關LaJolla鈦鍺精品的滿意度:價錢、品牌形象、製作材料(與特性)、其他顧客反應、商品認證保證、售後服務、交貨服務、整體商品的價值、品牌知名度、設計感
\end{enumerate}
分析結果
本研究採用網路與百貨公司附近發放方式,針對可能聽過鈦鍺精品的人進行問卷調查供調查603份剔除重複填寫與漏填,有效問卷599份 
本研究的樣本資料分析結果顯示 
\begin{enumerate}
\item 男性佔 50.4 % 女性 49.6% 如表 \ref{tab:PL1} 所示 
\item 年齡分為18歲以下 9.7% 、19 ~25歲 72.0% 、26~35歲 9.8%、36~40歲2.7% 、40歲以上5.8%本問卷族群與年輕族群居多。如表  \ref{tab:PL2} 所示
\item 教育方面 高中以下4.1%、高中(職)19.2%、專科5.3%、大學63.7%、碩士5.3%、博士1.7% 教育程度以大學64.7%為最多 。如表 \ref{tab:PL3} 所示
\item 工作性質 學生52.8%、服務業29.5%、製造業3.5%、軍公教4.3%、自由業8.5%、管家1.3%由職業可看以學生64.7%為最高。如表 \ref{tab:PL4} 所示
\end{enumerate}

\begin{table}[htb]
\caption{敘述統計(性別)}
\label{tab:PL1}
\renewcommand{\arraystretch}{1.2} % 將表格行間距加大為原來的 1.2 倍
\arrayrulewidth=1pt               % 調整線條粗細為 1pt
\tabcolsep=60pt                   % 調整欄間距為 24pt
%\begin{document}
\begin{tabular}[t]{lll}  % 第一欄位使用 sans serif 字族
\hline
 性別&次數 & 百分比 \\
\hline
男生        & 302 & 50.4 \\
女生        & 297  & 49.6 \\
總和        & 599  & 100 \\
\hline
\centering
\label{fig:PL4}
\end{tabular}
\end{table}

\begin{table}[htb]
\caption{敘述統計(年齡)}
\label{tab:PL2}
\renewcommand{\arraystretch}{1.2} % 將表格行間距加大為原來的 1.2 倍
\arrayrulewidth=1pt               % 調整線條粗細為 1pt
\tabcolsep=60pt                   % 調整欄間距為 24pt
%\begin{document}
\begin{tabular}[t]{lll}  % 第一欄位使用 sans serif 字族
\hline
 年齡& 次數 & 百分比 \\
\hline
18歲以下        & 58  & 9.7 \\
19~25歲        & 431  & 72.0 \\
26~35歲        & 59  & 9.8 \\
36~40歲        & 16  &2.7\\
40歲以上        & 35  & 5.8 \\
總和               & 599  & 100 \\
\hline
\end{tabular}
\end{table}

\begin{table}[htb]
\caption{敘述統計(學歷)}
\label{tab:PL3}
\renewcommand{\arraystretch}{1.2} % 將表格行間距加大為原來的 1.2 倍
\arrayrulewidth=1pt               % 調整線條粗細為 1pt
\tabcolsep=60pt                   % 調整欄間距為 24pt
%\begin{document}
\begin{tabular}[t]{lll}  % 第一欄位使用 sans serif 字族
\hline
 學歷& 次數 & 百分比 \\
\hline
高職以下       & 25  & 4.2 \\
高中(職)        & 116  &19.4\\
專科        & 32  & 5.3 \\
大學        & 384 &64.1\\
碩士        & 32  & 5.3 \\
博士           & 10  & 1.7 \\
總和           & 599  & 100 \\
\hline
\end{tabular}
\end{table}

\begin{table}[htb]
\caption{敘述統計(工作性質)}
\label{tab:PL4}
\renewcommand{\arraystretch}{1.2} % 將表格行間距加大為原來的 1.2 倍
\arrayrulewidth=1pt               % 調整線條粗細為 1pt
\tabcolsep=60pt                   % 調整欄間距為 24pt
\begin{tabular}[t]{lll}  % 第一欄位使用 sans serif 字族
\hline
 工作性質& 次數 & 百分比 \\
\hline
學生           & 316  & 52.8 \\
服務業        & 177  & 29.5 \\
製造業        & 21  & 3.5 \\
軍公教        & 26  &4.3\\
自由業        & 51  & 8.5 \\
家管           & 8  & 1.3 \\
總和               & 599  & 100 \\
\hline
\end{tabular}
\end{table}

首先對樣本收集已說明,其次對本研究的變數衡量做信度分析最後驗證假設
信度分析:已進行實證分析針對問卷問項進行信度分析用來得知問卷設計所測得的結果是否有信度與穩定性本研究採用目前已研究最常使用的 信賴度數做信度量測指標 Nunnally(1978)認為信度0.7以上表示高信度 可接受值大於0.7

本研究共有 3個變數,信度分析結果顯示
\begin{enumerate}
\item 行費者行為Cronbach's Alpha 值 0.946  如表 \ref{tab:e1}  所示
\item 品牌知覺Cronbach's Alpha 值 0.936  如表 \ref{tab:e2}  所示
\item 品牌聲望Cronbach's Alpha 值  0.958 如表 \ref{tab:e3}  所示
\item 消費者滿意度Cronbach's Alpha 值 0.969 如表 \ref{tab:e4}  所示
\end{enumerate}
以上信度 Alpha 值 大於 0.7以上顯示本研究的變數具有不錯的可信度。

\begin{table}[htb]
\caption{信度 (消費者行為)}
\label{tab:e1}
\renewcommand{\arraystretch}{1.2} % 將表格行間距加大為原來的 1.2 倍
\arrayrulewidth=1pt               % 調整線條粗細為 1pt
\tabcolsep=18pt                   % 調整欄間距為 24pt
\begin{tabular}[t]{lllll}  % 第一欄位使用 sans serif 字族
\hline
  & 代號& 平均數 & 標準差&  Alpha 值  \\
\hline
消費者行為&行為1&3.9873&0.82930&0.945735\\
               &行為2&3.9429	&0.82742&\\	
               &行為3&3.9460&0.81796&\\
               &行為4&3.9143&0.85713&\\
               &行為5&4.0540&0.82955&\\
               &行為6&4.0476&0.84515&\\
               &行為7&3.9905&0.83508&\\
               &行為8&4.0413&0.83792&\\
               &行為9&3.8381&0.86094&\\
               &行為10&3.9905&0.86136&\\
\hline
\end{tabular}
\end{table}

\begin{table}[htb]
\caption{信度 (品牌知覺)}
\label{tab:e1}
\renewcommand{\arraystretch}{1.2} % 將表格行間距加大為原來的 1.2 倍
\arrayrulewidth=1pt               % 調整線條粗細為 1pt
\tabcolsep=18pt                   % 調整欄間距為 24pt
\begin{tabular}[t]{lllll}  % 第一欄位使用 sans serif 字族
\hline
 & 代號& 平均數 & 標準差&  Alpha 值  \\
\hline
品牌知覺 & 知覺1  & 3.7753   & 0.7479  &0.936242 \\
              & 知覺2  & 3.7483  &0.79497  &  \\
             & 知覺3  & 3.8345    & 0.80373  &  \\
             & 知覺4 & 3.7500   &0.75513  &\\
             & 知覺5   & 3.7534  & 0.77393 &  \\
             & 知覺6  & 3.7534    &  0.77393 & \\
             & 知覺7  & 3.6909     & 0.80231  &  \\
\hline
\end{tabular}
\end{table}


\begin{table}[htb]
\caption{信度 (品牌聲望)}
\label{tab:e3}
\renewcommand{\arraystretch}{1.2} % 將表格行間距加大為原來的 1.2 倍
\arrayrulewidth=1pt               % 調整線條粗細為 1pt
\tabcolsep=18pt                   % 調整欄間距為 24pt
\begin{tabular}[t]{lllll}  % 第一欄位使用 sans serif 字族
\hline
 & 代號& 平均數 & 標準差&  Alpha 值  \\
\hline
品牌知覺 & 聲望1  & 3.6835 &0.82500 &0.958532\\
              & 聲望2  & 3.7089 &0.83439 &  \\
             & 聲望3  &3.6962 &0.85266  &  \\
             & 聲望4  &3.5949&0.75987\\
             & 聲望5  & 3.7468 &0.83924 &  \\
             & 聲望6  & 3.6456&0.78508& \\
             & 聲望7  & 3.6709&0.77900  &  \\
             & 聲望8  & 3.7215&0.84636&  \\
             & 聲望9  &3.6456&0.81709&  \\
             & 聲望10  & 3.7848&0.82696 &  \\
\hline
\end{tabular}
\end{table}

\begin{table}[htb]
\caption{信度 (消費者滿意度)}
\label{tab:e4}
\renewcommand{\arraystretch}{1.2} % 將表格行間距加大為原來的 1.2 倍
\arrayrulewidth=1pt               % 調整線條粗細為 1pt
\tabcolsep=18pt                   % 調整欄間距為 24pt
\begin{tabular}[t]{lllll}  % 第一欄位使用 sans serif 字族
\hline
 & 代號& 平均數 & 標準差&  Alpha 值  \\
\hline
品牌知覺 & 滿意度1&3.233&0.97143&0.969024\\
              & 滿意度2&3.4667&1.10589&  \\
             & 滿意度3&3.5000&1.00858&  \\
             & 滿意度4&3.5667&1.13512&\\
             & 滿意度5&3.7000&1.08755&  \\
             & 滿意度6&3.5667&1.16511&\\
             & 滿意度7&3.4667&1.00801&  \\
             & 滿意度8&3.6333&0.99943&  \\
             & 滿意度9&3.4333&1.04000&\\
             & 滿意度10&3.5667&1.07265&\\
\hline
\end{tabular}
\end{table}

因素分析
為了探討受訪者對主要考量因素,因此提出10提消費者行為、8題品牌知覺、10題品牌知覺、10題消費者滿意度等變數以量表收集受訪者對每一變數之重視度(非常不同意=1、非常同意=5)。將所獲得之資料,經過KMO取樣適當性及巴氏球形檢定。
\begin{enumerate}
\item 消費者行為
KMO=0.948 大於0.9表示分析效果極佳 Bartlett 的球形檢值 2427.156 顯著性.000<α = 0.01 顯示資料非常適合因素分析  本部分特性值大於1之標準將10個變數濃縮為1個因變數(主成分)全部變異可解釋為67.400%,詳細的數據如表  \ref{tab:p4} 所示。
\item 品牌知覺
KMO=0.926 大於0.9表示分析效果極佳 Bartlett 的球形檢值 3210.519 顯著性.000<α = 0.01 顯示資料非常適合因素分析  本部分特性值大於1之標準將7個變數濃縮為1個因變數(主成分)全部變異可解釋為72.457%,詳細的數據如表  \ref{tab:p4} 所示。
\item 品牌聲望
KMO=0.899 大於0.8表示分析有價值 Bartlett 的球形檢值 800.032 顯著性.000<α= 0.01 顯示資料非常適合因素分析  本部分特性值大於1之標準將10個變數濃縮為1個因變數(主成分)全部變異可解釋為73.020% 如圖 \ref{tab:p4}  所示。
\item 消費者滿意度
KMO=0.788 大於0.7表示分析中等 Bartlett 的球形檢值 420.646 顯著性.000<0.01 顯示資料非常適合因素分析  本部分特性值大於1之標準將10個變數濃縮為1個因變數(主成分)全部變異可解釋為78.632% 如圖 \ref{tab:p4} 所示
\end{enumerate}

\begin{table}[htb]
\caption{因數分析}
\label{tab:p4}
\renewcommand{\arraystretch}{1.2} % 將表格行間距加大為原來的 1.2 倍
\arrayrulewidth=1pt               % 調整線條粗細為 1pt
\tabcolsep=18pt                   % 調整欄間距為 24pt
\begin{tabular}[t]{llll}  % 第一欄位使用 sans serif 字族
\hline
 $因素分析$& $KMO$ & $Bartlett$& $全部異變數$ \\
\hline
消費者行為&0.948&2427.156&67.399863\\
 品牌知覺&0.926&3210.519&72.457307  \\
 品牌聲望&0.899&800.032&73.020478  \\
消費者滿意度&0.788&420.646&78.632077  \\
\hline
\end{tabular}
\end{table}

%\begin{figure}[!t]
%\centering
%\includegraphics[width=8cm]{images/kom.PNG}
%\caption{因素分析}
%\label{fig:p4}
%\end{figure}

假設之驗證
本研究共有6個假設待驗證,均採用簡單迴歸分析,結果整理於表 \ref{fig:p7} 所示以下對每個假設驗證結果加以說明
\begin{enumerate}
\item H1.品牌知覺消與費者行為有顯著的關係。
以簡單迴歸分析得值為:簡單相關數(需要填)判定數(R平方)為0.502調過的R平方數為0.501,滿意度與聲望迴歸值為0.665其t值17.572顯著性值為=0.000<α=0.05,結果為棄卻因變數與自變數間無迴歸關係存在之虛無假設;自變數與因變數間有存在直線關係,因此可看出品牌知覺與品牌聲望有顯著的關係之假設=成立 \ref{tab:r01}  所示。。
\item H2.品牌聲望與消費者行為有顯著的關係
以簡單迴歸分析得值為:簡單相關數(需要填) 判定數(R平方)為0.239調過的R平方數為0.228,滿意度與聲望迴歸值為0.501其t值4.610 顯著性值為=0.000<α=0.05,結果為棄卻因變數與自變數間無迴歸關係存在之虛無假設;自變數與因變數間有存在直線關係,因此可看出品牌知覺與品牌聲望有顯著的關係之假設=成立。 \ref{tab:r02}  所示。
\item H3.品牌聲望與品牌知覺有顯著的關係。
以簡單迴歸分析得值為:簡單相關數0.480 判定數(R平方)為0.231調過的R平方數為0.220,滿意度與聲望迴歸值為0.492其t值4.678 顯著性值為=0.000<α=0.05,結果為棄卻因變數與自變數間無迴歸關係存在之虛無假設;自變數與因變數間有存在直線關係,因此可看出品牌知覺與品牌聲望有顯著的關係之假設=成立。如表 \ref{tab:r1}  所示。
\item H4.品牌知覺與消費者滿意度有顯著的關係。
以簡單迴歸分析得值為:意度與知覺為0.060其t值0.647 顯著性值為=0.524<α=0.05結果為無法棄卻因變數與自變數間無迴關關係純在之虛無假設;自變數與因變數間無存在直線關係因此可看出品牌聲望與消費者滿意度有顯著的關係之假設=不成立。如表 \ref{tab:r3}  所示。
\item H5.品牌聲望與消費者滿意度有顯著的關係。
以簡單迴歸分析得值為:簡單相關數為0.893 判定數(R平方)為0.798調過的R平方數為0.780,滿意度與聲望迴歸值為0.692其t值7.832 顯著性值為=0.000<α=0.05, 結果為棄卻因變數與自變數間無迴歸關係存在之虛無假設;自變數與因變數間有存在直線關係,因此可看出品牌知覺與消費者滿意度有顯著的關係之假設=成立。如表 \ref{tab:r2}  所示。
\item H6.消費者行為與消費者滿意度有顯著的關係
以簡單迴歸分析得值為:簡單相關數(需要填) 判定數(R平方)為0.280調過的R平方數為0.252,滿意度與聲望迴歸值為0.423 其t值3.176 顯著性值為=0.004<α=0.05,結果為棄卻因變數與自變數間無迴歸關係存在之虛無假設;自變數與因變數間有存在直線關係,因此可看出品牌知覺與品牌聲望有顯著的關係之假設=成立。 \ref{tab:r03}  所示。
\end{enumerate}


\begin{table}[htb]
\caption{假設}
\label{tab:p7}
\renewcommand{\arraystretch}{1.2} % 將表格行間距加大為原來的 1.2 倍
\arrayrulewidth=1pt               % 調整線條粗細為 1pt
\tabcolsep=10pt                   % 調整欄間距為 24pt
\begin{tabular}[t]{llllll}  % 第一欄位使用 sans serif 字族
\hline
 假設&模型路徑&B之估計值& P值& t& 結果 \\
\hline
H1&品牌知覺→消費者行為&0.665&0.000&17.572&成立\\
H2&品牌聲望→消費者行為&0.501&0.000&4.619&成立\\
H3&品牌聲望→品牌知覺&0.490&0.000&4.648&成立\\
H4&品牌知覺→消費者滿意度&0.060&0.524&0.647&不成立\\
H5&品牌聲望→消費者滿意度&0.692&0.000&7.832&成立\\
H5&消費者行為→消費者滿意度&0.423&0.004&3.176&成立\\
\hline
\end{tabular}
\end{table}

\begin{table}[htb]
\caption{簡單回歸(依變數:行為的構面)}
\label{tab:r01}
\renewcommand{\arraystretch}{1.2} % 將表格行間距加大為原來的 1.2 倍
\arrayrulewidth=1pt               % 調整線條粗細為 1pt
\tabcolsep=10pt                   % 調整欄間距為 24pt
\begin{tabular}[t]{llllll}  % 第一欄位使用 sans serif 字族
\hline
 模型&B估計值&標準誤差&Beta分配&t&顯著性\\
\hline
(常數)&-.077&0.040& &-1.949&0.052\\
知覺的構面&0.665&0.038&0.709&17.572&0.000\\
\hline
\end{tabular}
\end{table}

\begin{table}[htb]
\caption{簡單回歸(依變數:行為的構面)}
\label{tab:r02}
\renewcommand{\arraystretch}{1.2} % 將表格行間距加大為原來的 1.2 倍
\arrayrulewidth=1pt               % 調整線條粗細為 1pt
\tabcolsep=10pt                   % 調整欄間距為 24pt
\begin{tabular}[t]{llllll}  % 第一欄位使用 sans serif 字族
\hline
 模型&B估計值&標準誤差&Beta分配&t&顯著性\\
\hline
(常數)&-.108&0.111& &-.975&0.333\\
聲望的構面&0.501&0.108&0.489&4.619&0.000\\
\hline
\end{tabular}
\end{table}

\begin{table}[htb]
\caption{簡單回歸(依變數:知覺構面)}
\label{tab:r1}
\renewcommand{\arraystretch}{1.2} % 將表格行間距加大為原來的 1.2 倍
\arrayrulewidth=1pt               % 調整線條粗細為 1pt
\tabcolsep=10pt                   % 調整欄間距為 24pt
\begin{tabular}[t]{llllll}  % 第一欄位使用 sans serif 字族
\hline
 模型&B估計值&標準誤差&Beta分配&t&顯著性\\
\hline
(常數)&0.249&0.105&&2.375&0.020\\
聲望的構面&0.490&0.105&0.480&4.648&0.000\\
\hline
\end{tabular}
\end{table}

\begin{table}[htb]
\caption{簡單回歸(依變數:滿意度構面)}
\label{tab:r2}
\renewcommand{\arraystretch}{1.2} % 將表格行間距加大為原來的 1.2 倍
\arrayrulewidth=1pt               % 調整線條粗細為 1pt
\tabcolsep=10pt                   % 調整欄間距為 24pt
\begin{tabular}[t]{llllll}  % 第一欄位使用 sans serif 字族
\hline
 模型&B估計值&標準誤差&Beta分配&t&顯著性\\
\hline
(常數)&0.158&0.103&&1.531&0.140\\
聲望的構面&0.692&0.088&0.857&7.832&0.000\\
\hline
\end{tabular}
\end{table}

\begin{table}[htb]
\caption{簡單回歸(依變數:滿意度構面)}
\label{tab:r3}
\renewcommand{\arraystretch}{1.2} % 將表格行間距加大為原來的 1.2 倍
\arrayrulewidth=1pt               % 調整線條粗細為 1pt
\tabcolsep=10pt                   % 調整欄間距為 24pt
\begin{tabular}[t]{llllll}  % 第一欄位使用 sans serif 字族
\hline
 模型&B估計值&標準誤差&Beta分配&t&顯著性\\
\hline
(常數)&0.158&0.103&&1.531&0.140\\
知覺的構面&0.060&0.093&0.071&0.647&0.524\\
\hline
\centering
\end{tabular}
\end{table}

\begin{table}[htb]
\caption{簡單回歸(依變數:滿意的構面)}
\label{tab:r03}
\renewcommand{\arraystretch}{1.2} % 將表格行間距加大為原來的 1.2 倍
\arrayrulewidth=1pt               % 調整線條粗細為 1pt
\tabcolsep=10pt                   % 調整欄間距為 24pt
\begin{tabular}[t]{llllll}  % 第一欄位使用 sans serif 字族
\hline
 模型&B估計值&標準誤差&Beta分配&t&顯著性\\
\hline
(常數)&0.092&0.173& &0.531&0.600\\
行為的構面&0.423&0.133&0.529&3.176&0.004\\
\hline
\end{tabular}
\end{table}


\chapter{結論與建議}

結論請分成下列幾點依序說明。

\section{結論}
\blindtext[1]


\section{未來發展建議}

\blindtext[1]

\section{後續研究方向}

\blindtext[1]

%\chapter{研究生必讀:如何交代你的「研究方法」}

%在論文中如何交代你的「研究方法」?

%在論文中,「研究方法」一節必須回答以下兩個問題:一是資料是如何搜集到或者衍生出來的,二是資料是如何分析的。換句話說,你要告訴讀者你是怎樣得到你的研究結果的。

%但是為甚麼你須要解釋你怎樣得到你的研究結果呢?不習慣現代西洋人學術思考方式的華人研究者,對此也許會感到困惑不解。他們在撰寫研究計劃或者論文本文時經常不曉得要寫些甚麼,尤其不曉得為甚麼要花那麼大的力氣去交代研究過程,好像過程比結果更重要 ──結果不對而過程對可以是好論文,反而結果對而過程不對就不是好論文,沒有瞎貓抓到死老鼠這種好事。以下姑且依據西洋人對學術論文的可能想法,列出六點理由略加說明。這六點理由同時亦間接指示了我們應該往甚麼方向去交代我們的研究方法:

%\begin{enumerate}[1., noitemsep]
%\item 因為研究方法會影響到研究結果。譬如說,如果你在探討台北市捷運乘客對台北市捷運效能的看法,而你所使用的是可作多項選擇的問卷而不是對個別乘客進行訪談,那麼你所得到的結果就會有所不同。讀者若知道你用甚麼方法得到資料,即有助於他評估你的研究結果是否有效和可信。

%\item 同一個研究問題可以有多種探究方法,你必須交代為何你決定選用某種方法來研究,而捨棄別的相關方法。
%\item 讀者也想知道你得到資料的方式是否合於探究主題之常理。譬如說,如果你使用問卷調查法來探究台北市捷運乘客對於捷運效能的看法,而你的問卷中卻只提供「1.非常好2.很好3.好」等三種正向選擇的回答而完全不提供負向選擇的回答,即不合於此種探究主題之常理。
%\item 讀者也想判斷你所使用的研究方法是否協合於研究目標。譬如在上述所例舉的探究中,你只個案研究一位乘客,這顯然就不協合於研究目標。
%\item 你也必須談一下你如何預防本來預期會發生的問題,而假如真有問題發生時你又如何把問題的衝擊力道減到最低。
%\item 如果你交代得夠清楚的話,有時候別的研究者會直接採用或者變相借用你的研究方法論(研究方法交代方式),特別是你的研究方法論頗有新意,又或者借用者慧眼獨具之時。
%\end{enumerate}


%至於在全篇論文中又該在甚麼地方和以甚麼方式交代你的研究方法呢?當然最重要的是「研究方法」這一章節了。在這一章節中你必須集中而扼要地說明你的研究方法。不過,並不是在這一章節中交代過就算了事了。在論文的其他章節中,也須適時酌量附帶說明一下。以下依論文章節安排略為提點一下:

%\begin{itemize}[noitemsep]
%\item 【緒論】研究問題介紹、研究目標介紹、研究程序介紹、主要研究結果與結論之選擇性介紹
%\item 【文獻回顧】回顧與你的研究問題有關的已有文獻(針對它們如何界定、說明和正當化此一研究問題而談)、回顧已有的相關研究方法論文獻(同樣針對它們如何界定、說明和正當化而談)、回顧已有的相關研究結果文獻(特別針對可信度等來談)
%\item 【研究方法】對獲取資料的方式和分析資料的方式詳加解釋、對方法論問題及其解決辦法和效果加以說明
%\item 【研究結果及討論】研究結果之呈現、關於研究結果之詮釋與延伸討論(譬如跟已有的研究結果作比較等)
%\item 【結論】說明研究問題是否有所「解答」、在何種程度上本研究達成了目標、我們從此研究結果中學到甚麼、從此研究結果中所學到的知識有甚麼用、此研究有何缺點等等
%\end{itemize}

%一般初學的研究者在交代研究方法時常犯有以下幾種錯誤:
%\begin{enumerate}[1., noitemsep]
%\item 無關的細節講太多。
%\item 不必要的初學者方法知識和程序細節仍然交代得清清楚楚,徒增篇幅。
%\item 忽略了討論搜集資料時所遇到的問題,以及沒有交代如何克服困難和權衡輕重時的考量點。
%\end{enumerate}

%以下介紹一些比較常見的研究類型:

%\begin{enumerate}[1., noitemsep]

%\item 【個案研究】(case study)對於一個或多個個人、團體、社群、企業或機構之背景、現況、環境和發展歷程予以觀察、記錄、分析,就其內部和外部的諸種影響而言得出某些階段性的變化模式來。
%\item 【比較研究】(comparative study)比較兩個或多個情況之間的異同。
%【相關與預測的研究】(correlation-prediction study)求得一些因素之間在統計上有顯著意義的相關係數,並且加以詮釋,以作為預測未來類同情況之參考。
%\item 【評估研究】(evaluation study)判斷某種計劃或安排是否遵循預定的程序並且達成其明說的目標。
%\item 【設計與展示的研究】(design-demonstration study)建構、測試和評估新的體系或新的程式是否可行。
%\item 【實驗研究】(experimental study)對控制其中一項或多項變異因素所求得的結果加以分析。
%\item 【問卷調查研究】(survey-questionnaire study)以問卷方式對某個特定團體的行為、信念和意見予以確定、報導並詮釋。
%\item 【狀態特性研究】(status study)對於一個或多個現象之代表性案例或者取樣案例加以觀察檢驗,以確定其特殊性格。
%\item 【理論建構研究】(theory construction study)尋找或描述一些可解釋事物之所以如此運作的原理。
%\item 【趨勢分析研究】(trend analysis study)分析目前事件之動力結構以便預測或預斷事件之未來走向。
%\item 【概念分析研究】(concept analysis study)對於某些學術上使用的關鍵性概念作語意方面或邏輯方面的分析。
%\end{enumerate}

%\input{chap4}

%\input{chap_last}

%%% 參考文獻
\newpage
\baselineskip=20pt
\renewcommand{\baselinestretch}{1}
\phantomsection % for hyperref to register this
\addcontentsline{toc}{chapter}{\nameRef}
\renewcommand{\bibname}{\protect\makebox[5cm][s]{\nameRef}}
%  \makebox{} is fragile; need protect
\bibliographystyle{ieeetr}  % 使用 IEEE Trans 期刊格式
%\bibliography{my_bib}

\begin{thebibliography}{1}

%\bibitem{IEEEhowto:kopka}
%H.~Kopka and P.~W. Daly, \emph{A Guide to \LaTeX}, 3rd~ed.\hskip 1em plus 0.5em minus 0.4em\relax Harlow, England: Addison-Wesley, 1999.

\bibitem{AMA}
American Marketing Association,AMA http://www.marketingpower.com/

\bibitem{Sappington}
Sappington, D. E. M. &  Wernerfelt, B. (1985). To brand or not to brand? A theoretical and e、mpirical question. Journal of Business, 58(3), 279-293.

\bibitem{Farquhar}
Farquhar, P. H. (1989). Managing brand equity. Journal of Marketing Research, 30, 7-12.

\bibitem{Rao}
Rao, A. R., & Ruekert, R. W. (1994). Brand alliances as signals of product quality. Sloan Management Review, 36(1), 87-97.

\bibitem{Boyd}
Boyd, H. W., Walker, O. C., & Larreche, J. C. (1995). Marketing management a
strategic approach with a global orientation. (2nd ed), Boston: Irwin Inc.

\bibitem{Laforet}
Laforet, S., & Saunders, J. (1994). Managing brand portfolios: Theleaders do it. Journal of Advertising Research, 64-76.

\bibitem{Aaker1991}
Aaker, D. A. (1991), Managing Brand Equity: Capitalizing on the Value of a Brand Name, New
York: The Free Press
\bibitem{Aaker1996}
Aaker, D. A. 1996, “Measuring brand equity across products and markets.”California Management Review, 38, No.3, pp. 102-20.

\bibitem{Aaker1998}
Keller, K. L. (1998), Strategic Brand Management: Building, Measuring, and Managing
Brand Equity, NJ: Prentice-Hall Press.

\bibitem{Aaker1990.47}
Aaker, D. A. "Brand Extensions: The Good, the Bad, and the Ugly”, Sloan 
Management Review, (Summer), 1990, pp. 47-56. 

\bibitem{Aaker1990}
Aaker, D. A.,  & Keller, K. L. “
Consumer Evaluation of Brand Extension”, 
Journal of Marketing, 54(January), 1990, pp. 27-41.

\bibitem{Pratt1974}
Pratt, Robert W. (1974). Measuring purchase Behavior in Handbook of Marketing. Robert Ferber, N.Y.Mc Graw Hill Inc.

\bibitem{Keller1993}
Keller, K. L. (1993), “Conceptualizing, measuring, and managing cu
brand equity”, Journal of Marketing, 57, 2, 1-22. 
\bibitem{Boyd1995}
Boyd,H.W.Jr., Walker. O.C. Jr. &Larreche. Jean-Claude(1995), “Maketing Mangerment a Strategic Approach with a Global Orientation,“2th ed.,pp.25-31
\bibitem{Keller1998}
Keller, K. L. (1998), Strategic Brand Management: Building, Measuring, and Managing
Brand Equity, NJ: Prentice-Hall Press.

\bibitem{Engel2001}
Engel, J.F., R.D. Blackwell, and P.W. Miniard (2001), Consumer Behavior , 9th
ed., Harcourt College Publishers.
\bibitem{Bulter1998}
Bulter Patrick, and Joe Peppard, “Consumer Purchasing on the Internet: Proceses and Prospects,” European Management Journal, Vol. 16, Iss. 15, pp:600-610
\bibitem{Chernatonyl2001}
De Chernatony, L. (2001). From Brand Vision to Brand Evaluation: Strategically Building and Sustaining Brands. Butter worth-Heinemann.

\bibitem{Walter1970}
Walter, C. G.,&Gordon, P. W. (1970). Consumer behaviors: an integrated framework. Homewood, IL: Irwin.

\bibitem{Oliver1977}
Oliver, R. L. (1977). Effect of expectation and disconfirmation on postexposure product
evaluations: An alternative interpretation. Journal of Applied Psychology, 62 (4)
\bibitem{Oliver1980}
Oliver, R. L. (1980). A cognitive model of the antecedents and consequences of satisfaction
decisions. Journal of Marketing Research, 17 (4), 460–469.
\bibitem{Oliver1981}
Oliver, Richard L. (1981). Measureemnt and evaluation of satisfaction processes in retailing setting. Journal
\bibitem{Zellwegger}
Zellwegger, P.(1997), “Web-based sales,” Electronic Markets 7, pp:10-161
\bibitem{Cardozo1964}
Cardozo, R. N. (1964). “Customer Satisfaction: Laboratory Study and Marketing Action”. Journal of Marketing Research, 2, 244-249.
\bibitem{Chernatony2001}
De Chernatony, L. (2001). From Brand Vision to Brand Evaluation: Strategically Building and Sustaining Brands. Butter worth-Heinemann.
\bibitem{Chernatony1998}
De Chernatony, L., McDonald, M. (1998). Creating Powerful Brands in Consumer, Services and Industrial Markets (2 Ed.) Oxford: Butterworth-Heinemann.
\bibitem{Nunnally}
Nunnally, J.C., (1978), Psychometric Theory, New York: McGraw-Hill. 

\bibitem{Turner2011}
Turner, A. (2011). The New “It” Metric. Retrieved from

\bibitem{EBL}
Engel, James F., Roger D. Blackwell, and David T. Kollat (1993), Consumer
Behavior, 7th edition, Chicago Dryden Press.

\bibitem{資策會}
資訊工業策進會-http://www.iii.org.tw/
\bibitem{汪澤普}
汪澤普 (2005)實體與線上零售通路間的銷售掠奪: 以線上雜誌為例
\bibitem{林靈宏,張魁峯}
林靈宏,張魁峯 ,(2009), 消費者行為學(Consumer Behavior), 6頁
\bibitem{鄒風、孟林明}
鄒風、孟林明,消費心理學,世界商業文庫,民國86年。
\bibitem{汪澤普}
汪澤普(2005)實體與線上零售通路間的銷售掠奪: 以線上雜誌為例
\bibitem{王翎幗}
王翎幗(2006) B2C電子商務網站成功之預測模式
\bibitem{林靈宏}
林靈宏 (2000),消費者行為學,台北:五南圖書出版公司。
\bibitem{Schiffman}
Schiffman, L. G.,  & Kanuk, L. L.著,顧萱萱、郭建志譯 (2003),消費者行為,台北: 學富文化。
\bibitem{周宗霖}
周宗霖(2012)網路消費平台回復性服務品質對其消費 者滿意度及忠誠度影響程度之研究 -以網路運動商品消費者為例
\bibitem{楊聰林}
楊聰林(2005),「顧客滿意與顧客忠誠之實證研究-以臺灣某一物流公司為例」,朝
  陽科技大學工業工程與管理系碩士論文。
\bibitem{藍悅真}
藍悅真(2012)美利達品牌形象對消費者滿意度及忠誠度之影響
\bibitem{蔡曜光}
蔡曜光(2011)品牌形象與消費者涉入對購買行為之研究-以連鎖咖啡館為例
\bibitem{林素吟}
林素吟(2005)產品保證與品牌知名度對顧客忠誠度的影響之研究 2005 年 2 月第 8 卷 1 期 • Vol. 8, No. 1, Feb 2005
\bibitem{程信賢}
程信賢(2002)行動電話消費者購買行為及其市場區隔之研究─以南部地區為例
\bibitem{陳思璇}
陳思璇(2011)網路商店品質對消費者滿意、信任及再購買意願之影響
\bibitem{張倫嚴}
張倫嚴(2012)產品創新、產品涉入、知覺價值與消費者滿意度之關聯性研究─以韓國3C產品為例
\bibitem{藍悅真}
藍悅真(2012)美利達品牌形象對消費者滿意度及忠誠度之影響
\bibitem{張定邦}
張定邦(2012) 大學生對研究所品牌知覺影響因素之研究
\end{thebibliography}
\clearpage
%\end{CJK}
\end{document}
