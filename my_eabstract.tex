%Although the global economy is facing recession worries, the sales strength of smart phone did not see slow down. The global handset shipments reach 1.6 billion sets in 2011, annual growth rate of only 11\% (excluding white box cell phone) the smart phone shipments of 450 million, a growth rate of more than 60\%, penetration rate of 27.95\%, the Topology Research Institute. 
      
%Because the driven both of emerging market demand and the parity trend commodities, expected 2012 growth momentum continued in smart mobile phone, the shipments approach 600 hundred million mark, the penetration rate of more than one-third predicted that by 2015, half of the world mobile phones are for the world of smart phones. 

%However when smart phones has NFC function, except to grasp whether the existing use of smart phones life would more convenient or not, to pay attention to the degree of the user are expected, or are still under observation of impact. 

%In this study, the Technology Acceptance Model (TAM), perceived usefulness and perceived ease of use questionnaire architectural foundation for the two influencing factors, coupled with the demand for mobile phones (dependence), the acceptance of new technology (satisfaction) and an additional fee on the use of NFC applications to an acceptable level of the three dimensions of variables. To understand the extent of changes in consumer behavior to use the Smartphone’s NFC function, and the SPSS 19.0 statistical analysis software to conduct analysis to deal with hypothesis testing. 

%The results showed that the perceived usefulness and perceived ease of use, level of demand for mobile phones (dependence), acceptance of new technology acceptance (satisfaction) and the use of NFC applications at an additional cost, all will positive impact on consumer behavioral intention to use the NFC smart phones.

%Keywords: mobile communication, NFC (Near field communication), smart phones, the technology acceptance model (TAM), theory of consumer behavior.
