% 使用 utf-8 編碼
% v2.0 (Apr. 5, 2009)
% 填入你的論文題目、姓名等資料
% 如果題目內有必須以數學模式表示的符號,請用 \mbox{} 包住數學模式,如下範例
% 如果中文名字是單名,與姓氏之間建議以全形空白填入,如下範例
% 英文名字中的稱謂,如 Prof. 以及 Dr.,其句點之後請以不斷行空白~代替一般空白,如下範例
% 如果你的指導教授沒有如預設的三位這麼多,則請把相對應的多餘教授的中文、英文名
%    的定義以空的大括號表示
%    如,\renewcommand\advisorCnameB{}
%          \renewcommand\advisorEnameB{}
%          \renewcommand\advisorCnameC{}
%          \renewcommand\advisorEnameC{}

% 論文題目 (中文)
\renewcommand\cTitle{%
品牌聲望及品牌知覺對消費決策為與消費者滿意度影響之研究-以飾品業 La Jolla公司為例(論文初稿)}

% 論文題目 (英文)
\renewcommand\eTitle{
A study of the Influences of Brand perception and Brand Reputation on consumer decision making and Consumer satisfaction – A case study of La Jolla jewelry company}

% 我的姓名 (中文)
\renewcommand\myCname{盧建璋}

% 我的姓名 (英文)
\renewcommand\myEname{jian-Jhang Lu}

% 指導教授A的姓名 (中文)
\renewcommand\advisorCnameA{林慶昌 博士}

% 指導教授A的姓名 (英文)
\renewcommand\advisorEnameA{Dr.~Ching-Chang Lin}

% 指導教授B的姓名 (中文)
\renewcommand\advisorCnameB{}

% 指導教授B的姓名 (英文)
\renewcommand\advisorEnameB{}

% 指導教授C的姓名 (中文)
\renewcommand\advisorCnameC{}

% 指導教授C的姓名 (英文)
\renewcommand\advisorEnameC{}

% 校名 (中文)
\renewcommand\univCname{臺北城市科技大學}

% 校名 (英文)
\renewcommand\univEname{Taipei Chengshih University of Science and Technology}

% 系所名 (中文)
\renewcommand\deptCname{電子商務研究所}

% 系所全名 (英文)
\renewcommand\fulldeptEname{Institute of E-Commerce}

% 系所短名 (英文, 用於書名頁學位名領域)
\renewcommand\deptEname{Institute of E-Commerce}

% 學院英文名 (如無,則以空的大括號表示)
\renewcommand\collEname{College of Business and Management}

% 學位名 (中文)
\renewcommand\degreeCname{碩士}

% 學位名 (英文)
\renewcommand\degreeEname{Master}

% 口試年份 (中文、民國)
\renewcommand\cYear{一百零二}

% 口試月份 (中文)
\renewcommand\cMonth{六}

% 口試日 (中文)
\renewcommand\cDay{二十六}

% 口試年份 (阿拉伯數字、西元)
\renewcommand\eYear{2013}

% 口試月份 (英文)
\renewcommand\eMonth{June}

% 口試日 (英文)
\renewcommand\eDay{26}


% 學校所在地 (英文)
\renewcommand\ePlace{Taipei, Taiwan}

%畢業級別;用於書背列印;若無此需要可忽略
\newcommand\GraduationClass{100}

%%%%%%%%%%%%%%%%%%%%%%
