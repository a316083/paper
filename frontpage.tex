%
% 使用 utf-8 編碼
% v2.0 (Apr. 5, 2009)
% do not change the content of this file
% unless the thesis layout rule is changed
% 無須修改本檔內容,除非校方修改了
% 封面、書名頁、中文摘要、英文摘要、誌謝、目錄、表目錄、圖目錄、符號說明
% 等頁之格式

% make the line spacing in effect
\renewcommand{\baselinestretch}{\mybaselinestretch}
\large % it needs a font size changing command to be effective

% default variables definitions
% 此處只是預設值,不需更改此處
% 請更改 my_names.tex 內容
\newcommand\cTitle{論文題目}
\newcommand\eTitle{MY THESIS TITLE}
\newcommand\myCname{王鐵雄}
\newcommand\myEname{Aron Wang}
\newcommand\advisorCnameA{南宮明博士}
\newcommand\advisorEnameA{Dr.~Ming Nangong}
\newcommand\advisorCnameB{李斯坦博士}
\newcommand\advisorEnameB{Dr.~Stein Lee}
\newcommand\advisorCnameC{徐 石博士}
\newcommand\advisorEnameC{Dr.~Sean~Hsu}
\newcommand\univCname{臺北城市大學}
\newcommand\univEname{Taipei Chengshih University of Science and Technology}
\newcommand\deptCname{電子商務研究所}
\newcommand\fulldeptEname{Institute of E-Commerce}
\newcommand\deptEname{Institute of E-Commerce}
\newcommand\collEname{College of Business and Management}
\newcommand\degreeCname{碩士}
\newcommand\degreeEname{Master}
\newcommand\cYear{九十四}
\newcommand\cMonth{六}
\newcommand\cDay{}
\newcommand\eYear{2006}
\newcommand\eMonth{June}
\newcommand\eDay{}
\newcommand\ePlace{Taipei, Taiwan}


 % user's names; to replace those default variable definitions
% 使用 utf-8 編碼
% v2.0 (Apr. 5, 2009)
% 填入你的論文題目、姓名等資料
% 如果題目內有必須以數學模式表示的符號,請用 \mbox{} 包住數學模式,如下範例
% 如果中文名字是單名,與姓氏之間建議以全形空白填入,如下範例
% 英文名字中的稱謂,如 Prof. 以及 Dr.,其句點之後請以不斷行空白~代替一般空白,如下範例
% 如果你的指導教授沒有如預設的三位這麼多,則請把相對應的多餘教授的中文、英文名
%    的定義以空的大括號表示
%    如,\renewcommand\advisorCnameB{}
%          \renewcommand\advisorEnameB{}
%          \renewcommand\advisorCnameC{}
%          \renewcommand\advisorEnameC{}

% 論文題目 (中文)
\renewcommand\cTitle{%
品牌聲望及品牌知覺對消費決策為與消費者滿意度影響之研究-以飾品業 La Jolla公司為例(論文初稿)}

% 論文題目 (英文)
\renewcommand\eTitle{
A study of the Influences of Brand perception and Brand Reputation on consumer decision making and Consumer satisfaction – A case study of La Jolla jewelry company}

% 我的姓名 (中文)
\renewcommand\myCname{盧建璋}

% 我的姓名 (英文)
\renewcommand\myEname{jian-Jhang Lu}

% 指導教授A的姓名 (中文)
\renewcommand\advisorCnameA{林慶昌 博士}

% 指導教授A的姓名 (英文)
\renewcommand\advisorEnameA{Dr.~Ching-Chang Lin}

% 指導教授B的姓名 (中文)
\renewcommand\advisorCnameB{}

% 指導教授B的姓名 (英文)
\renewcommand\advisorEnameB{}

% 指導教授C的姓名 (中文)
\renewcommand\advisorCnameC{}

% 指導教授C的姓名 (英文)
\renewcommand\advisorEnameC{}

% 校名 (中文)
\renewcommand\univCname{臺北城市科技大學}

% 校名 (英文)
\renewcommand\univEname{Taipei Chengshih University of Science and Technology}

% 系所名 (中文)
\renewcommand\deptCname{電子商務研究所}

% 系所全名 (英文)
\renewcommand\fulldeptEname{Institute of E-Commerce}

% 系所短名 (英文, 用於書名頁學位名領域)
\renewcommand\deptEname{Institute of E-Commerce}

% 學院英文名 (如無,則以空的大括號表示)
\renewcommand\collEname{College of Business and Management}

% 學位名 (中文)
\renewcommand\degreeCname{碩士}

% 學位名 (英文)
\renewcommand\degreeEname{Master}

% 口試年份 (中文、民國)
\renewcommand\cYear{一百零二}

% 口試月份 (中文)
\renewcommand\cMonth{六}

% 口試日 (中文)
\renewcommand\cDay{二十六}

% 口試年份 (阿拉伯數字、西元)
\renewcommand\eYear{2013}

% 口試月份 (英文)
\renewcommand\eMonth{June}

% 口試日 (英文)
\renewcommand\eDay{26}


% 學校所在地 (英文)
\renewcommand\ePlace{Taipei, Taiwan}

%畢業級別;用於書背列印;若無此需要可忽略
\newcommand\GraduationClass{100}

%%%%%%%%%%%%%%%%%%%%%%


% 使用 hyperref 在 pdf 簡介欄裡填入相關資料
\ifx\hypersetup\undefined
	\relax  % do nothing
\else
	\hypersetup{
	pdftitle=\cTitle,
	pdfauthor=\myCname}
\fi
	

\newcommand\itsempty{}
%%%%%%%%%%%%%%%%%%%%%%%%%%%%%%%
%       TPCU cover 封面
%%%%%%%%%%%%%%%%%%%%%%%%%%%%%%%
%
\begin{titlepage}
% no page number
% next page will be page 1

% aligned to the center of the page
\begin{center}
% font size (relative to 12 pt):
% \large (14pt) < \Large (18pt) < \LARGE (20pt) < \huge (24pt)< \Huge (24 pt)
%
\makebox[6cm][s]{\Huge{\univCname}}\\  %顯示中文校名
\vspace{1.5cm}
\makebox[12cm][s]{\Huge{\deptCname}}\\ %顯示中文系所名
\vspace{1.5cm}
\makebox[6cm][s]{\Huge{\degreeCname 論文}}\\ %顯示論文種類 (中文)
% 初稿論文
%\vspace{1.5cm}\makebox[2cm][s]{\Large{(初稿)}}\\ %顯示論文種類 (中文)
\vspace{3cm}
%
% Set the line spacing to single for the titles (to compress the lines)
\renewcommand{\baselinestretch}{1}   %行距 1 倍
%\large % it needs a font size changing command to be effective
%\Large{\cTitle}\\  % 中文題目
\fontsize{20}{22}\selectfont{\cTitle} \\
%
\vspace{1cm}
%
%\Large{\eTitle}\\ %英文題目
\fontsize{20}{22}\selectfont{\eTitle} \\

\vspace{4cm}
% \makebox is a text box with specified width;
% option s: stretch
% use \makebox to make sure
% 「研究生:」 與「指導教授:」occupy the same width
\hspace{4.5cm} \makebox[3cm][s]{\Large{研 究 生:}}
\Large{\myCname}  % 顯示作者中文名
\hfill \makebox[1cm][s]{}\\
%
\vspace{1cm}
\hspace{4.5cm} \makebox[3cm][s]{\Large{指導教授:}}
\Large{\advisorCnameA}  %顯示指導教授A中文名
\hfill \makebox[1cm][s]{}\\
%
% 判斷是否有共同指導的教授 B
\ifx \advisorCnameB  \itsempty
\relax % 沒有 B 教授,所以不佔版面,不印任何空白
\else
% 共同指導的教授 B
\hspace{4.5cm} \makebox[3cm][s]{}
\Large{\advisorCnameB}  %顯示指導教授B中文名
\hfill \makebox[1cm][s]{}\\
\fi
%
% 判斷是否有共同指導的教授 C
\ifx \advisorCnameC  \itsempty
\relax % 沒有 C 教授,所以不佔版面,不印任何空白
\else
% 共同指導的教授 C
\hspace{4.5cm} \makebox[3cm][s]{}
\Large{\advisorCnameC}  %顯示指導教授C中文名
\hfill \makebox[1cm][s]{}\\
\fi
%
\vfill
\makebox[10cm][s]{\Large{中華民國\cYear 年\cMonth 月\cDay 日}}%
%
\end{center}
% Resume the line spacing to the desired setting
\renewcommand{\baselinestretch}{\mybaselinestretch}   %恢復原設定
% it needs a font size changing command to be effective
% restore the font size to normal
\normalsize
\end{titlepage}
%%%%%%%%%%%%%%

%% 從摘要到本文之前的部份以小寫羅馬數字印頁碼
% 但是從「書名頁」(但不印頁碼) 就開始計算
\setcounter{page}{1}
\pagenumbering{roman}
%%%%%%%%%%%%%%%%%%%%%%%%%%%%%%%
%       書名頁
%%%%%%%%%%%%%%%%%%%%%%%%%%%%%%%
%
\newpage

% 判斷是否要浮水印?
\ifx\mywatermark\undefined
  \thispagestyle{empty}  % 無頁碼、無 header (無浮水印)
\else
  \thispagestyle{EmptyWaterMarkPage} % 無頁碼、有浮水印
\fi

%no page number
% create an entry in table of contents for 書名頁
\phantomsection % for hyperref to register this
\addcontentsline{toc}{chapter}{\nameInnerCover}


% aligned to the center of the page
\begin{center}
% font size (relative to 12 pt):
% \large (14pt) < \Large (18pt) < \LARGE (20pt) < \huge (24pt)< \Huge (24 pt)
% Set the line spacing to single for the titles (to compress the lines)
\renewcommand{\baselinestretch}{1}   %行距 1 倍
% it needs a font size changing command to be effective
%中文題目
\Large{\cTitle}\\ %%%%%
\vspace{1cm}
% 英文題目
\Large{\eTitle}\\ %%%%%
%\vspace{1cm}
\vfill
% \makebox is a text box with specified width;
% option s: stretch
% use \makebox to make sure
% 「研究生:」 與「指導教授:」occupy the same width
\makebox[3cm][s]{\large{研 究 生:}}
\makebox[3cm][l]{\large{\myCname}} %%%%%
\hfill
\makebox[2cm][s]{\large{Student: }}
\makebox[5cm][l]{\large{\myEname}}\\ %%%%%
%
%\vspace{1cm}
%
\makebox[3cm][s]{\large{指導教授:}}
\makebox[3cm][l]{\large{\advisorCnameA}} %%%%%
\hfill
\makebox[2cm][s]{\large{Advisor: }}
\makebox[5cm][l]{\large{\advisorEnameA}}\\ %%%%%
%
% 判斷是否有共同指導的教授 B
\ifx \advisorCnameB  \itsempty
\relax % 沒有 B 教授,所以不佔版面,不印任何空白
\else
%共同指導的教授B
\makebox[3cm][s]{}
\makebox[3cm][l]{\large{\advisorCnameB}} %%%%%
\hfill
\makebox[2cm][s]{}
\makebox[5cm][l]{\large{\advisorEnameB}}\\ %%%%%
\fi
%
% 判斷是否有共同指導的教授 C
\ifx \advisorCnameC  \itsempty
\relax % 沒有 C 教授,所以不佔版面,不印任何空白
\else
%共同指導的教授C
\makebox[3cm][s]{}
\makebox[3cm][l]{\large{\advisorCnameC}} %%%%%
\hfill
\makebox[2cm][s]{}
\makebox[5cm][l]{\large{\advisorEnameC}}\\ %%%%%
\fi
%
% Resume the line spacing to the desired setting
\renewcommand{\baselinestretch}{\mybaselinestretch}   %恢復原設定
\large %it needs a font size changing command to be effective
%
\vfill
\makebox[4cm][s]{\large{\univCname}}\\% 校名
\makebox[6cm][s]{\large{\deptCname}}\\% 系所名
\makebox[3cm][s]{\large{\degreeCname 論文}}\\% 學位名
%
%\vspace{1cm}
\vfill
\large{A Thesis}\\%
\large{Submitted to }%
%
\large{\fulldeptEname}\\%系所全名 (英文)
%
%
\ifx \collEname  \itsempty
\relax % 沒有學院名 (英文),所以不佔版面,不印任何空白
\else
% 有學院名 (英文)
\large{\collEname}\\% 學院名 (英文)
\fi
%
\large{\univEname}\\%校名 (英文)
%
\large{in Partial Fulfillment of the Requirements}\\
%
\large{for the Degree of}\\
%
\large{\degreeEname}\\%學位名(英文)
%
\large{in}\\
%
\large{\deptEname}\\%系所短名(英文;表明學位領域)
%
\large{\eMonth\ \eYear}\\%月、年 (英文)
%
\large{\ePlace}% 學校所在地 (英文)
\vfill
\makebox[10cm][s]{\Large{中華民國\cYear 年\cMonth 月\cDay 日}}
%\large{中華民國}%
%\large{\cYear}% %%%%%
%\large{年}%
%\large{\cMonth}% %%%%%
%\large{月}\\
\end{center}
% restore the font size to normal
\normalsize
\clearpage
%%%%%%%%%%%%%%%%%%%%%%%%%%%%%%%
%       論文口試委員審定書 (計頁碼,但不印頁碼)
%%%%%%%%%%%%%%%%%%%%%%%%%%%%%%%
%
% insert the printed standard form when the thesis is ready to bind
% 在口試完成後,再將已簽名的審定書放入以便裝訂
% create an entry in table of contents for 審定書
% 目前送出空白頁
\newpage%
%\begin{CJK}{UTF8}{nkai}
\thispagestyle{EmptyWaterMarkPage}  % 無 header,但在浮水印模式下會有浮水印
\phantomsection % for hyperref to register this
\addcontentsline{toc}{chapter}{\nameCommitteeForm}%

\begin{center}
\fontsize{24}{36}\selectfont 碩士學位論文考試委員會審定書\\
\vspace{0.5cm}
\end{center}

\vspace{0.5cm}

%\parbox[s]
%{\cTitle} \\
%{\eTitle} \\

\fontsize{20}{22}\selectfont
本校 \underline{\deptCname} \underline{\myCname}  君\\
所提論文 \uline{\cTitle} \\
%本校 \underline{\deptCname} \underline{\myCname}  君,所提論文 \\
%\fontsize{18}{22}\selectfont\cTitle \\
%\fontsize{20}{22}\selectfont
經本委員會審定通過,合於碩士資格,特此證明。 \\

%\vspace{0.1cm}
\fontsize{20}{16}\selectfont
\begin{tabular}{rp{0.9\textwidth}}
學位考試委員會 \\ & \\
委~員~:
	& \rule{0.6\textwidth}{1pt} \\
	& \\
	& \rule{0.6\textwidth}{1pt} \\
	& \\
	& \rule{0.6\textwidth}{1pt} \\
	& \\
	& \rule{0.6\textwidth}{1pt} \\
	& \\
指~導~教~授~: & \rule{0.6\textwidth}{1pt} \\
	& \\
所~長~: & \rule{0.6\textwidth}{1pt}
\end{tabular}

%\vfill
\vspace{1cm}
\begin{center}
\makebox[10cm][s]{\Large{中華民國\cYear 年\cMonth 月\cDay 日}}
\end{center}
%\large{中華民國}
%\large{\cYear}% %%%%%
%\large{年}
%\large{\cMonth}% %%%%%
%\large{月}
%\large{\cDay}% %%%%%
%\large{日}\\

%\end{CJK}
\clearpage
\normalsize


%%%%%%%%%%%%%%%%%%%%%%%%%%%%%%%
%       論文口試委員審定書 (計頁碼,但不印頁碼)
%%%%%%%%%%%%%%%%%%%%%%%%%%%%%%%
%
% insert the printed standard form when the thesis is ready to bind
% 在口試完成後,再將已簽名的審定書放入以便裝訂
% create an entry in table of contents for 審定書
% 目前送出空白頁
%%\newpage%
%%{\thispagestyle{empty}%
%%\phantomsection % for hyperref to register this
%%\addcontentsline{toc}{chapter}{\nameCommitteeForm}%
%%\mbox{}\clearpage}

%%%%%%%%%%%%%%%%%%%%%%%%%%%%%%%
%       授權書 (計頁碼,但不印頁碼)
%%%%%%%%%%%%%%%%%%%%%%%%%%%%%%%
%
% insert the printed standard form when the thesis is ready to bind
% 在口試完成後,再將已簽名的授權書放入以便裝訂
% create an entry in table of contents for 授權書
% 目前送出空白頁
\newpage%

%\vspace{3cm}
此頁為「論文授權書」,計頁碼,但不印頁碼,

\vspace{3cm}
在口試完成後,再將已簽名的授權書之後,替換本頁以便裝訂。
{\thispagestyle{empty}%
\phantomsection % for hyperref to register this
\addcontentsline{toc}{chapter}{\nameCopyrightForm}%
\mbox{}\clearpage}

%%%%%%%%%%%%%%%%%%%%%%%%%%%%%%%
%       中文摘要
%%%%%%%%%%%%%%%%%%%%%%%%%%%%%%%
%
\newpage
\thispagestyle{plain}  % 無 header,但在浮水印模式下會有浮水印
% create an entry in table of contents for 中文摘要
\phantomsection % for hyperref to register this
\addcontentsline{toc}{chapter}{\nameCabstract}

% aligned to the center of the page
\begin{center}
% font size (relative to 12 pt):
% \large (14pt) < \Large (18pt) < \LARGE (20pt) < \huge (24pt)< \Huge (24 pt)
% Set the line spacing to single for the names (to compress the lines)
\renewcommand{\baselinestretch}{1}   %行距 1 倍
% it needs a font size changing command to be effective
\large{\cTitle}\\  %中文題目
\vspace{0.83cm}
% \makebox is a text box with specified width;
% option s: stretch
% use \makebox to make sure
% each text field occupies the same width
\makebox[1.5cm][s]{\large{學生:}}
\makebox[3cm][l]{\large{\myCname}} %學生中文姓名
\hfill
%
\makebox[3cm][s]{\large{指導教授:}}
\makebox[3cm][l]{\large{\advisorCnameA}} \\ %教授A中文姓名
%
% 判斷是否有共同指導的教授 B
\ifx \advisorCnameB  \itsempty
\relax % 沒有 B 教授,所以不佔版面,不印任何空白
\else
%共同指導的教授B
\makebox[1.5cm][s]{}
\makebox[3cm][l]{} %%%%%
\hfill
\makebox[3cm][s]{}
\makebox[3cm][l]{\large{\advisorCnameB}}\\ %教授B中文姓名
\fi
%
% 判斷是否有共同指導的教授 C
\ifx \advisorCnameC  \itsempty
\relax % 沒有 C 教授,所以不佔版面,不印任何空白
\else
%共同指導的教授C
\makebox[1.5cm][s]{}
\makebox[3cm][l]{} %%%%%
\hfill
\makebox[3cm][s]{}
\makebox[3cm][l]{\large{\advisorCnameC}}\\ %教授C中文姓名
\fi
%
\vspace{0.42cm}
%
\large{\univCname}\large{\deptCname}\\ %校名系所名
\vspace{0.83cm}
%\vfill
\makebox[2.5cm][s]{\large{摘要}}\\
\end{center}
% Resume the line spacing to the desired setting
\renewcommand{\baselinestretch}{\mybaselinestretch}   %恢復原設定
%it needs a font size changing command to be effective
% restore the font size to normal
\normalsize
%%%%%%%%%%%%%
因近年來隨著時代變遷網際網路盛行,導致網路購物興起源於實體通路商紛紛轉型向虛擬通路,本以實體通路時消費者購買商品時,隨時都有服務人員在場介紹,推銷幫助消費者更能瞭解購買的商品瞭解所需商品,但是虛擬網際網路通路並沒有像實體通路,服務人員隨時在旁協助因此很多公司並無法有效的瞭解網路消費者所需的,對於消費者是否滿意或其販售的商品,因此導致無法猜測試是否服務到客戶更難以測試未來的消費行為.

本研究以 La Jolla 樂活雅鈦鍺精品公司為例,探討公司的經營在導入電子商務經營策略之後,品牌聲望與及品牌知覺對消費者行為與消費者滿意度影響之研究,並以敘述統計、信度分析、迴歸分析來驗證,並提出未來經營的建議。
 

%La Jolla 樂活雅鈦鍺精品公司一直保持實體店面的營運模式,但為了增加曝光率,採用網路行銷策略,創立官方網站,經營網路論壇以及加入各大網路通路,期望能提昇品牌知名度。

%本研究以 La Jolla 樂活雅鈦鍺精品公司為例,探討公司的經營在導入電子商務經營策略之後,品牌聲望與及品牌知覺對消費者行為與消費者滿意度影響之研究,並以敘述統計、信度分析、迴歸分析來驗證,並提出未來經營的建議。


%針對以上發現,提出本研究整合的數據為研究結論,並以提出建議,以便提高廠商對於網路消費者行為之文獻以便網路通路相關業者改進網路品牌行銷之策略的參考

關鍵字:品牌聲望、品牌知覺、消費者意願、品牌行銷、La jolla樂活雅


%%%%%%%%%%%%%%%%%%%%%%%%%%%%%%%
%       英文摘要
%%%%%%%%%%%%%%%%%%%%%%%%%%%%%%%
%
\newpage
\thispagestyle{plain}  % 無 header,但在浮水印模式下會有浮水印

% create an entry in table of contents for 英文摘要
\phantomsection % for hyperref to register this
\addcontentsline{toc}{chapter}{\nameEabstract}

% aligned to the center of the page
\begin{center}
% font size:
% \large (14pt) < \Large (18pt) < \LARGE (20pt) < \huge (24pt)< \Huge (24 pt)
% Set the line spacing to single for the names (to compress the lines)
\renewcommand{\baselinestretch}{1}   %行距 1 倍
%\large % it needs a font size changing command to be effective
\large{\eTitle}\\  %英文題目
\vspace{0.83cm}
% \makebox is a text box with specified width;
% option s: stretch
% use \makebox to make sure
% each text field occupies the same width
\makebox[2cm][s]{\large{Student: }}
\makebox[5cm][l]{\large{\myEname}} %學生英文姓名
\hfill
%
\makebox[2cm][s]{\large{Advisor: }}
\makebox[5cm][l]{\large{\advisorEnameA}} \\ %教授A英文姓名
%
% 判斷是否有共同指導的教授 B
\ifx \advisorCnameB  \itsempty
\relax % 沒有 B 教授,所以不佔版面,不印任何空白
\else
%共同指導的教授B
\makebox[2cm][s]{}
\makebox[5cm][l]{} %%%%%
\hfill
\makebox[2cm][s]{}
\makebox[5cm][l]{\large{\advisorEnameB}}\\ %教授B英文姓名
\fi
%
% 判斷是否有共同指導的教授 C
\ifx \advisorCnameC  \itsempty
\relax % 沒有 C 教授,所以不佔版面,不印任何空白
\else
%共同指導的教授C
\makebox[2cm][s]{}
\makebox[5cm][l]{} %%%%%
\hfill
\makebox[2cm][s]{}
\makebox[5cm][l]{\large{\advisorEnameC}}\\ %教授C英文姓名
\fi
%
\vspace{0.42cm}
\large{Submitted to }\large{\fulldeptEname}\\  %英文系所全名
%
\ifx \collEname  \itsempty
\relax % 如果沒有學院名 (英文),則不佔版面,不印任何空白
\else
% 有學院名 (英文)
\large{\collEname}\\% 學院名 (英文)
\fi
%
\large{\univEname}\\  %英文校名
\vspace{0.83cm}
%\vfill
%
\large{ABSTRACT}\\
%\vspace{0.5cm}
\end{center}
% Resume the line spacing the desired setting
\renewcommand{\baselinestretch}{\mybaselinestretch}   %恢復原設定
%\large %it needs a font size changing command to be effective
% restore the font size to normal
\normalsize
%%%%%%%%%%%%%
Although the global economy is facing recession worries, the sales strength of smart phone did not see slow down. The global handset shipments reach 1.6 billion sets in 2011, annual growth rate of only 11\% (excluding white box cell phone) the smart phone shipments of 450 million, a growth rate of more than 60\%, penetration rate of 27.95\%, the Topology Research Institute. 
      
Because the driven both of emerging market demand and the parity trend commodities, expected 2012 growth momentum continued in smart mobile phone, the shipments approach 600 hundred million mark, the penetration rate of more than one-third predicted that by 2015, half of the world mobile phones are for the world of smart phones. 

However when smart phones has NFC function, except to grasp whether the existing use of smart phones life would more convenient or not, to pay attention to the degree of the user are expected, or are still under observation of impact. 

In this study, the Technology Acceptance Model (TAM), perceived usefulness and perceived ease of use questionnaire architectural foundation for the two influencing factors, coupled with the demand for mobile phones (dependence), the acceptance of new technology (satisfaction) and an additional fee on the use of NFC applications to an acceptable level of the three dimensions of variables. To understand the extent of changes in consumer behavior to use the Smartphone’s NFC function, and the SPSS 19.0 statistical analysis software to conduct analysis to deal with hypothesis testing. 

The results showed that the perceived usefulness and perceived ease of use, level of demand for mobile phones (dependence), acceptance of new technology acceptance (satisfaction) and the use of NFC applications at an additional cost, all will positive impact on consumer behavioral intention to use the NFC smart phones.

Keywords: mobile communication, NFC (Near field communication), smart phones, the technology acceptance model (TAM), theory of consumer behavior.


%%%%%%%%%%%%%%%%%%%%%%%%%%%%%%%
%       誌謝
%%%%%%%%%%%%%%%%%%%%%%%%%%%%%%%
%
% Acknowledgment
\newpage
\chapter*{\protect\makebox[5cm][s]{\nameAckn}} %\makebox{} is fragile; need protect
\phantomsection % for hyperref to register this
\addcontentsline{toc}{chapter}{\nameAckn}
本論文能如期完成研究,要感謝爸爸與媽媽的教育之恩與不斷支持,與指導教授「林慶昌博士」老師許多教授細心與耐心不嫌棄的與我討論與指教、還有實驗室的同學、學長姐、學弟,以及不離不棄在我身邊陪伴不斷加油打氣快放棄時給我加油的女朋友和在職場工作與不斷給我鼓勵的職場同仁們、給予本人無窮無盡向前的動力,在過程中體會到論文並不是一兩天就能夠完成的事,因此規劃了長期的時間。從完全沒有頭緒到有,從開始收集資料、整理資料、發現問題、遇到瓶頸,雖然辛苦,但得到經驗是不會讓辛苦白費的。研究的成果固然重要,但更重要的是過程中思考的方法跟經驗、與同學間的互相討論、鼓勵、以及不怕艱難和追求進步的精神,這一段時間漸漸使我養成獨立思考的能力,這兩年的時間與大家的互相學習、指教,是我們這輩子不會忘記的。

在碩士班的日子裡,實驗室就像是我第二個家,在這裡面結識的夥伴是在我人生中幫助我成長不可或缺的重要人物,這些日子裡陪伴著我一起成長。感謝碩士班裡的同學們雖然我們班人數最少,但我們也是最團結、最會團結合作的一個班不管是做報告,互相討論題目研究方向等都是一起努力的,還有剛進來讀碩士班的我學長姊都把我當同學、好朋友的照顧,我不懂的地方學長姊都會無私的給予我指導與解惑,碩士是我生涯中最重要的收獲,感謝所上給予我參與SPSS 數據分析的課程,在研習與準備證照中,從原本完全不懂分析的我到獲得認證過程中雖然非常的辛苦,但感謝培訓與給予我指導的老師們支持與鼓勵使得我終於獲得數據分析知識與認證。最重要的要感謝指導老師不離不棄與教導讓我學到很多例如編寫論文的XeLaTex 與GitHub 等許多的指導讓我在兩年來學習到對我未來有幫助的事情,未來我會學以致用不管是出社會在工作上或以後還有機會像上學習都會繼續研究都會盡心盡力,雖然求學的階段或許將告一段落,但是研究的路程是不會中斷的,非常感謝這段時間所給我的磨練,讓我的求學旅程中,因此更加豐富。在此獻上我最大的感謝。





%%%%%%%%%%%%%%%%%%%%%%%%%%%%%%%
%       目錄
%%%%%%%%%%%%%%%%%%%%%%%%%%%%%%%
%
% Table of contents
\newpage
\renewcommand{\contentsname}{\protect\makebox[5cm][s]{\nameToc}}
%\makebox{} is fragile; need protect
\phantomsection % for hyperref to register this
\addcontentsline{toc}{chapter}{\nameToc}
\tableofcontents

%%%%%%%%%%%%%%%%%%%%%%%%%%%%%%%
%       表目錄
%%%%%%%%%%%%%%%%%%%%%%%%%%%%%%%
%
% List of Tables
\newpage
\renewcommand{\listtablename}{\protect\makebox[5cm][s]{\nameLot}}
%\renewcommand{\numberline}[1]{\loflabel~#1\hspace*{1em}}
\renewcommand{\numberline}[1]{\lotlabel~#1\hspace*{1em}}   
%\makebox{} is fragile; need protect
\phantomsection % for hyperref to register this
\addcontentsline{toc}{chapter}{\nameLot}
\listoftables

%%%%%%%%%%%%%%%%%%%%%%%%%%%%%%%
%       圖目錄
%%%%%%%%%%%%%%%%%%%%%%%%%%%%%%%
%
% List of Figures
\newpage
\renewcommand{\listfigurename}{\protect\makebox[5cm][s]{\nameTof}}
\renewcommand{\numberline}[1]{\loflabel~#1\hspace*{1em}}
%\renewcommand{\numberline}[1]{\lotlabel~#1\hspace*{1em}}   
%\makebox{} is fragile; need protect
\phantomsection % for hyperref to register this
\addcontentsline{toc}{chapter}{\nameTof}
\listoffigures

%%%%%%%%%%%%%%%%%%%%%%%%%%%%%%%
%       符號說明
%%%%%%%%%%%%%%%%%%%%%%%%%%%%%%%
%
% Symbol list
% define new environment, based on standard description environment
% adapted from p.60~64, <<The LaTeX Companion>>, 1994, ISBN 0-201-54199-8
% 目前用不到
%\newcommand{\SymEntryLabel}[1]%
%  {\makebox[3cm][l]{#1}}
%%
%\newenvironment{SymEntry}
%   {\begin{list}{}%
%       {\renewcommand{\makelabel}{\SymEntryLabel}%
%        \setlength{\labelwidth}{3cm}%
%        \setlength{\leftmargin}{\labelwidth}%
%        }%
%   }%
%   {\end{list}}
%%%
%\newpage
%\chapter*{\protect\makebox[5cm][s]{\nameSlist}} %\makebox{} is fragile; need protect
%\phantomsection % for hyperref to register this
%\addcontentsline{toc}{chapter}{\nameSlist}
%\input{my_symbols.tex}


\newpage
%% 論文本體頁碼回復為阿拉伯數字計頁,並從頭起算
\pagenumbering{arabic}
%%%%%%%%%%%%%%%%%%%%%%%%%%%%%%%%
