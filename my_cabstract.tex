因近年來隨著時代變遷網際網路盛行,導致網路購物興起源於實體通路商紛紛轉型向虛擬通路,本以實體通路時消費者購買商品時,隨時都有服務人員在場介紹,推銷幫助消費者更能瞭解購買的商品瞭解所需商品,但是虛擬網際網路通路並沒有像實體通路,服務人員隨時在旁協助因此很多公司並無法有效的瞭解網路消費者所需的,對於消費者是否滿意或其販售的商品,因此導致無法猜測試是否服務到客戶更難以測試未來的消費行為.

本研究以 La Jolla 樂活雅鈦鍺精品公司為例,探討公司的經營在導入電子商務經營策略之後,品牌聲望與及品牌知覺對消費者行為與消費者滿意度影響之研究,並以敘述統計、信度分析、迴歸分析來驗證,並提出未來經營的建議。
 

%La Jolla 樂活雅鈦鍺精品公司一直保持實體店面的營運模式,但為了增加曝光率,採用網路行銷策略,創立官方網站,經營網路論壇以及加入各大網路通路,期望能提昇品牌知名度。

%本研究以 La Jolla 樂活雅鈦鍺精品公司為例,探討公司的經營在導入電子商務經營策略之後,品牌聲望與及品牌知覺對消費者行為與消費者滿意度影響之研究,並以敘述統計、信度分析、迴歸分析來驗證,並提出未來經營的建議。


%針對以上發現,提出本研究整合的數據為研究結論,並以提出建議,以便提高廠商對於網路消費者行為之文獻以便網路通路相關業者改進網路品牌行銷之策略的參考

關鍵字:品牌聲望、品牌知覺、消費者意願、品牌行銷、La jolla樂活雅
