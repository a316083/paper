本論文能如期完成研究,要感謝爸爸與媽媽的教育之恩與不斷支持,與指導教授「林慶昌博士」老師許多教授細心與耐心不嫌棄的與我討論與指教、還有實驗室的同學、學長姐、學弟,以及不離不棄在我身邊陪伴不斷加油打氣快放棄時給我加油的女朋友和在職場工作與不斷給我鼓勵的職場同仁們、給予本人無窮無盡向前的動力,在過程中體會到論文並不是一兩天就能夠完成的事,因此規劃了長期的時間。從完全沒有頭緒到有,從開始收集資料、整理資料、發現問題、遇到瓶頸,雖然辛苦,但得到經驗是不會讓辛苦白費的。研究的成果固然重要,但更重要的是過程中思考的方法跟經驗、與同學間的互相討論、鼓勵、以及不怕艱難和追求進步的精神,這一段時間漸漸使我養成獨立思考的能力,這兩年的時間與大家的互相學習、指教,是我們這輩子不會忘記的。

在碩士班的日子裡,實驗室就像是我第二個家,在這裡面結識的夥伴是在我人生中幫助我成長不可或缺的重要人物,這些日子裡陪伴著我一起成長。感謝碩士班裡的同學們雖然我們班人數最少,但我們也是最團結、最會團結合作的一個班不管是做報告,互相討論題目研究方向等都是一起努力的,還有剛進來讀碩士班的我學長姊都把我當同學、好朋友的照顧,我不懂的地方學長姊都會無私的給予我指導與解惑,碩士是我生涯中最重要的收獲,感謝所上給予我參與SPSS 數據分析的課程,在研習與準備證照中,從原本完全不懂分析的我到獲得認證過程中雖然非常的辛苦,但感謝培訓與給予我指導的老師們支持與鼓勵使得我終於獲得數據分析知識與認證。最重要的要感謝指導老師不離不棄與教導讓我學到很多例如編寫論文的XeLaTex 與GitHub 等許多的指導讓我在兩年來學習到對我未來有幫助的事情,未來我會學以致用不管是出社會在工作上或以後還有機會像上學習都會繼續研究都會盡心盡力,雖然求學的階段或許將告一段落,但是研究的路程是不會中斷的,非常感謝這段時間所給我的磨練,讓我的求學旅程中,因此更加豐富。在此獻上我最大的感謝。



